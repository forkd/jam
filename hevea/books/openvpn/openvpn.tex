Virtual Private Network (VPN) é a utilização de uma rede de comunicações de forma privada. Através da criptografia cria-se uma espécie de túnel seguro por onde trafegam as informações. Em resumo, é a utilização de um meio inseguro no transporte seguro de informações.

Pode ligar duas redes, um computador a uma rede ou dois computadores.

VPN é uma ferramenta de segurança e pode ser aplicada a um conjunto de boas práticas, como descrito nas ISOs 17799 e 27001.

Conceitos

Túnel: consiste em encapsular um protocolo dentro de outro. No caso das VPNs atuais, este encapsulamento inclui a criptografia.
Protocolos: na camada 2 (enlace) há o Point-to-Point Tunneling Protocol (PPTP) da Microsoft, o Layer 2 Tunneling Protocol (L2TP) da IETF e o Layer 2 Forwarding (L2F) da Cisco. Na camada 3 (rede), há o IP Security Tunnel Mode (IPSec) da IETF, que foi desenvolvido com suporte ao IPv6 e muitos profissionais o consideram o padrão do futuro.

OpenVPN: software livre, multiplataforma. Cria VPNs tipo SSL (VPN SSL), com o OpenSSL. Não é compatível com o IPSec e opera com a arquitetura cliente-servidor. Trabalha de preferência com o protocolo UDP, com portas acima de 5000. Utiliza duas interfaces de conexão: na camada 3, a TUN e na camada 2, a TAP.

No Linux, o kernel deve ter sido compilado com suporte a dispositivos TUN/TAP e a biblitoteca de compressão LZO pode estar presente no sistema.

Configuração

Em ambos (servidor e cliente), recomenda-se criar o diretório /etc/openvpn para concentrar os arquivos de configuração. No servidor, dentro deste diretório, criar a chave de criptografia:

# openvpn --genkey --secret server-client.asc

Será criado o arquivo server-client.asc, que deverá ser copiado para o cliente. Ainda no servidor, criar o arquivo server.conf:

proto udp
dev tun
ifconfig <ip-gw-vpn-server> <ip-gw-vpn-client>
secret /etc/openvpn/server-client.asc
port 5900

No cliente este arquivo ficará:

proto udp
remote <ip-real-server>
dev tun
ifconfig <ip-gw-vpn-client> <ip-gw-vpn-server>
secret /etc/openvpn/server-client.asc
port 5900

Executar o OpenVPN em cada máquina:

# openvpn --config /etc/openvpn/server.conf &
# openvpn --config /etc/openvpn/client.conf &

Com isso surgirá a nova NIC, tun0. Para testar a conexão, basta "pingar" nos endereços dados ao gateway da VPN, no servidor e no cliente. Agora configure as rotas no servidor e no cliente:

# route add -net <subnet-server> netmask <mask> gw <ip-gw-vpn-client>

# route add -net <subnet-client> netmask <mask> gw <ip-gw-vpn-server>

Liberar acesso no firewall:

# iptables -t nat -A POSTROUTING -o tun+ -j MASQUERADE
# iptables -A FORWARD -i tun+ -s <subnet-server>/<mask> -d <subnet-client>/<mask> -j ACCEPT

# iptables -t nat -A POSTROUTING -o tun+ -j MASQUERADE
# iptables -A FORWARD -i tun+ -s <subnet-client>/<mask> -d <subnet-server>/<mask> -j ACCEPT

---

O OpenVPN pode trabalhar com nenhuma criptografia (sem sentido para a maioria dos casos), criptografia com chaves estáticas (mostrado) e em modo Transport Layer Security (TLS) com chaves dinâmicas. O TLS é considerado o sucessor do Secure Sockets Layer (SSL) e permite a implementação disso.

Para implementar, primeiro deve-se configurar o arquivo /etc/openssl/openssl.cnf. Na opção dir, indicar /etc/openvpn. Na opção default_days, indicar por quantos dias a chave será válida. Em /etc/openvpn, criar os arquivos index.txt (vazio) e serial, com o valor 01.

Dentro de /etc/openvpn, criar a autoridade certificadora (CA):

# openssl req -nodes -new -x509 -keyout ca.asc -out ca.crt

Gerar a chave Diffie-Hellman:

# dhparam -out dh.pem

Gerar o pedido de certificado do servidor:

# openssl req -nodes -new -keyout server.asc -out server.csr

Assinar o pedido:

# openssl ca -out server.crt -in server.csr

Pode-se descartar o arquivo do pedido (server.csr), ficando com:

ca.crt - CA
dh.pem - Diffie-Hellman
server.crt - chave privada do servidor
server.asc - chave pública do servidor

Com isso o /etc/openvpn/server.conf ficaria:

dev tun
proto udp
ifconfig <ip-gw-vpn-server> <ip-gw-vpn-client>
tls-server
dh /etc/openvpn/dh.pem
ca /etc/openvpn/ca.crt
cert /etc/openvpn/server.crt
key /etc/openvpn/server.asc
up /etc/openvpn/server-up.sh  # shell script executado na inicialização da VPN
down /etc/openvpn/server-down.sh  # shell script executado na finalização da VPN
port 5900
verb 3

Feito isso, deve-se executar o OpenVPN no servidor. Dica: a opção --daemon executa o software como um serviço.

Ainda no servidor, criar as chaves para o cliente:

# openssl req -nodes -new -keyout client.asc -out client.csr
# openssl ca -out client.crt -in client.csr

Copiar os arquivos client.asc, client.crt, dh.pem e ca.crt para o cliente. Depois configurar o /etc/openvpn/client.conf:

proto udp
remote <ip-real-server>
dev tun
tls-client
ifconfig <ip-gw-vpn-client> <ip-gw-vpn-server>
ca /etc/openvpn/ca.crt
cert /etc/openvpn/client.crt
key /etc/openvpn/client.asc
up /etc/openvpn/client-up.sh
down /etc/openvpn/client-down.sh
port 5900
verb 3

Caso não se tenha um endereço IP real fixo para o servidor, e.g., Velox, pode-se usar um serviço de DNS dinâmico para tal e indicar o domínio no parâmetro remote.

Todos os parâmetros passados no arquivo de configuração são acessíveis via linha de comando, mas pelo arquivo é mais legível e elegante.

No arquivo de configuração pode-se determinar a rota da VPN com:

route <ip-server|client> netmask <mask>
route-gateway <ip-gw>

Para utilizar um ambiente mais seguro pode-se abrir mão do chroot.

1. Criar o /var/empty se ele não existir e passar sua propriedade para nobody.nogroup:
# mkdir /var/empty && chown -R nobody.nogroup /var/empty

2. Configurar o servidor:

proto udp
dev tun
ifconfig <ip-gw-vpn-server> <ip-gw-vpn-client>
route <subnet-client> netmask <mask>
route-gateway <ip-gw-vpn-client>
secret /etc/openvpn/server.asc
port 5900
comp-lzo
tun-mtu 1500
fragment 1300
mssfix
ping 15
ping-restart 60
resolv-entry 3600
persist-key  # permite usar o daemon com usuários sem privilégios
user nobody
group nogroup
chroot /var/empty
