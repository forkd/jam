% Hevea - ITILv3 Resumo
% An ITILv3 overview.
%
% Author: José Lopes de Oliveira Jr. <jilo.cc>
%
% LICENSE
% This program is free software: you can redistribute it and/or modify
% it under the terms of the GNU General Public License as published by
% the Free Software Foundation, either version 3 of the License, or
% (at your option) any later version.
%
% This program is distributed in the hope that it will be useful,
% but WITHOUT ANY WARRANTY; without even the implied warranty of
% MERCHANTABILITY or FITNESS FOR A PARTICULAR PURPOSE.  See the
% GNU General Public License for more details.
%
% You should have received a copy of the GNU General Public License
% along with this program.  If not, see <http://www.gnu.org/licenses/>.
%%


\chapter{Desenho de Serviço}
\label{cha:design}


\begin{figure}
    \centering
    \includegraphics[width=0.5\textwidth]{img/peony}\\
    {\scriptsize Peony --- X-Ray Arts}
\end{figure}

É o desenho de serviços de TI apropriados e invadores, incluindo suas
arquiteturas, processos, políticas e documentações, para suprir requisitos de
negócio atuais e futuros. Desenha serviços que são desenvolvidos dentro de uma
escala de tempo e custo, respeitando os ANSs.


\section{Conceitos}
\label{sec:design:conc}
\begin{itemize}
    \item Provedor de Serviço: é a entidade responsável pela entrega de um
        serviço aos clientes e negócio.
    \item Fornecedor: são os prestadores de serviço externos.
    \item ANS: o Acordo de Nível de Serviço descreve os serviços de TI que o
        provedor deve entregar e os níveis de serviço requeridos. Horário de
        atendimento, tempo de resposta e percentual de disponibilidade são
        algumas da metas que podem ser estabelecidas.
    \item ANO: para que o provedor de serviço consiga cumprir as metas
        estabelecidas nos ANS com o cliente, é necessário que ele estabeleça
        metas internas com suas equipes técnicas. Estas metas internas são
        chamadas de Acordo de Nível Operacional.
    \item Contrato: é relevante que todos os serviços entregues por terceiros
        estejam baseados em um contrato ---também conhecido como Contrato de
        Apoio---, pois ele tem um valor legal entre as partes. Os acordos não
        têm valor legal, são apenas documentos que homologam as metas entre as
        partes.
    \item Pacote de Desenho de Serviço: é uma documentação do projeto para cada
        novo serviço, mudança de grande impacto, remoção de um serviço ou
        mudança em um Pacote de Desenho de Serviço.
    \item Disponibilidade: é a proporção de tempo que um cliente está
        habilitado a acessar, quando requerido, um serviço específico ou a
        habilidade de executar uma função acordada. É mensurada a partir do
        ponto de vista do cliente e registrada no ANS. O tempo/percentual da
        disponibilidade de um serviço deve ser compatível com o tempo de
        serviço acordado.
\end{itemize}
\nomenclature{ANO}{Acordo de Nível Operacional}


\section{Os 4 Ps}
\label{sec:design:4ps}
\begin{itemize}
    \item Pessoas: é importante determinar os papéis das pessoas nos processos.
    \item Processos: devem ser bem definidos.
    \item Produtos: precisam ser determinados, inclusive serviços.
    \item Parceiros: devem ser estabelecidos
\end{itemize}

Os 4 Ps se interrelacionam e representam as competências necessárias que o
provedor de serviços deve ter.


\section{Algumas Opções de Fornecimento de Serviço}
\label{sec:design:ofs}
\begin{itemize}
    \item Insourcing: utiliza recursos internos à organização.
    \item Outsourcing: utiliza recursos externos à organização.
    \item Co-sourcing: combina in e outsourcing.
    \item Multi-sourcing: arranjo para organizações trabalharem em conjunto, as
        atividades são distribuídas por vários fornecedores.
\end{itemize}


\section{Processos}
\label{sec:design:processos}
\subsection{Gerenciamento de Nível de Serviço}
É responsável por garantir entendimento entre as necessidades dos clientes e o
que o provedor de TI deve entregar. Estabelecer acordos ---ANSs--- é uma forma
de gerenciar a expectativa do cliente, pois ele saberá o que poderá exigir do
provedor, que se beneficiará por passar a ter um claro entendimento do que deve
entregar.

Gerente de Nível de Serviço é o papel descrito neste processo e suas atividades
incluem, dentre outras, garantir que as metas de níveis de serviço estão
alinhadas com os ANSs, revisar acordos internos e externos e analisar a
satisfação do cliente.


\subsection{Gerenciamento do Catálogo de Serviço}
Proporciona um único local de informações consistentes sobre todos os serviços
acordados e assegura que o catálogo esteja disponível para quem tem autorização
de acessá-lo. O Catálogo de Serviço está inserido dentro do Portfolio de
Serviço, entretanto, este documento é bem mais estruturado e tem informações
detalhadas dos serviços.

O Gerente de Catálogo de Serviço é responsável por produzir e manter o
catálogo.


\subsection{Gerenciamento de Disponibilidade}
Tem como meta assegurar que os serviços sejam entregues dentro dos níveis
acordados. É completado por 2 níveis relacionados:
\begin{enumerate}
    \item Disponibilidade do Serviço, que envolve todos os aspectos da
        disponibilidade de um serviço; e
    \item Disponibilidade do Componente, que envolve aspectos da
        disponibilidade de um componente do serviço.
\end{enumerate}

A disponibilidade é envolvida por 4 aspectos:

\begin{itemize}
    \item Disponibilidade: habilidade de um serviço, componente ou item de
        configuração executar a função acordada quando requerida;
    \item Confiabilidade: medida de quanto tempo um serviço, componente ou item
        de configuração pode executar a função acordada sem interrupção;
    \item Sustentabilidade: rapidez que um serviço, componente ou item de
        configuração consegue ser restaurado para seu estado normal após uma
        falha; e
    \item Funcionalidade: habilidade de um fornecedor externo em atender os
        termos de seu contrato.
\end{itemize}

O Gerente de Disponibilidade garante que todos os novos serviços são desenhados
para entregar o nível de disponibilidade requerido pelo negócio, dentre outras
funções.


\subsection{Gerenciamento da Segurança da Informação}
Visa controlar a provisão de informações e evitar seu uso não autorizado. Seus
objetivos são compostos por:
\begin{itemize}
    \item Confidencialidade: o acesso à informação será feito de forma correta;
    \item Integridade: a informação será completa e precisa;
    \item Disponibilidade: a informação estará disponível quando requerida; e
    \item Autenticidade: as trocas de informação entre a organização e seus
        parceiros será confiável.
\end{itemize}

Controles de segurança precisam ser estabelecidos para atender aos requisitos
de entidades externas, como Banco Central e Sarbanes Oxley (SOX). A Política de
Segurança da Informação é o resultado deste processo.

Este documento estabelece um conjunto de controles que atendem os objetivos
internos à organização e regulamentos e leis externas. O Gerenciamento da
Segurança da Informação é baseado na ISO/IEC 27001 que estabelece as seguintes
etapas: Controlar, Planejar, Implantar, Avaliar e Manter.

O Gerente de Segurança é quem desenvolve e publica a Política de Segurança da
Informação na organização e garante que ela está sendo usada.


\subsection{Gerenciamento de Fornecedor}
Assegura que os fornecedores e os seus serviços são gerenciados para suportar
as metas dos serviços de TI e os ANSs. Assegura que contratos e acordos com
fornecedores estejam alinhados com as necessidades do negócio e com as metas
dos ANSs e ANOs, em conjunto com o Gerenciamento do Nível de Serviço.

O Banco de Dados de Contratos é um repositório central onde ficam os cadastros
de todos os fornecedores e os contratos relacionados. Dessa forma, os
fornecedores podem ser classificados conforme sua avaliação de risco e impacto:
\begin{itemize}
    \item Estratégicos: envolvem troca de informação confidencial;
    \item Táticos: envolvem atividades comerciais significativas;
    \item Operacionais: fornecem serviços ou produtos de operação; e
    \item Commodity: fornecedores de papel, tinta etc.
\end{itemize}

Cada fornecedor precisa de um tratamento diferenciado conforme sua importância.

Neste processo é definido o papel de Gerente de Fornecedor, que será
responsável por todas as tarefas descritas, catalogando, mantendo, avaliando,
comunicando e negociando contratos com fornecedores.


\subsection{Gerenciamento de Capacidade}
Assegura que a capacidade da infraestrutura de TI esteja alinhada com as
necessidades do negócio. Entende e mantém os níveis de entrega de serviços
resquisitados a um custo aceitável. Está continuamente tentando alcançar a
combinação de custo efetivo com os recursos de TI. É divido em 3 subprocessos:
\begin{itemize}
    \item Gerenciamento da Capacidade do Negócio: focado no longo prazo,
        assegura que os requisitos do negócio sejam levados em consideração;
    \item Gerenciamento da Capacidade de Serviço: assegura que a performance
        dos serviços estejam de acordo com os ANSs; e
    \item Gerenciamento da Capacidade de Componente: mais detalhado e técnico,
        otimiza a utilização dos recursos de hardware e software.
\end{itemize}

O Gerente de Capacidade atua com ponto focal para questões de capacidade e
desempenho, incluindo relatórios de gerenciamento sobre uso, tendências e
previsões.


\subsection{Gerenciamento da Continuidade do Serviço}
Prepara o provedor de serviço para a pior situação possível. Faz parte de um
processo maior, que não é de TI, mas sim da organização como um todo: o
Gerenciamento da Continuidade do Negócio.

Diferentemente do Gerenciamento da Disponibilidade, que foca em obter a melhor
disponibilidade possível para os serviços de TI em operação, o Gerenciamento da
Continuidade do Serviço foca em elaborar um plano de continuidade com
estratégias de recuperação para desastres.

O Gerente de Continuidade do Serviço criará e manterá o processo, assegurando
que ele esteja de acordo com o Gerenciamento de Continuidade do Negócio.
Realizará testes e ajudará na execução da análise de impacto do negócio para
todos os serviços.
