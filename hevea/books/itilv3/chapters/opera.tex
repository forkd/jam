% Hevea - ITILv3 Resumo
% An ITILv3 overview.
%
% Author: José Lopes de Oliveira Jr. <jilo.cc>
%
% LICENSE
% This program is free software: you can redistribute it and/or modify
% it under the terms of the GNU General Public License as published by
% the Free Software Foundation, either version 3 of the License, or
% (at your option) any later version.
%
% This program is distributed in the hope that it will be useful,
% but WITHOUT ANY WARRANTY; without even the implied warranty of
% MERCHANTABILITY or FITNESS FOR A PARTICULAR PURPOSE.  See the
% GNU General Public License for more details.
%
% You should have received a copy of the GNU General Public License
% along with this program.  If not, see <http://www.gnu.org/licenses/>.
%%


\chapter{Operação de Serviço}
\label{cha:opera}


\begin{figure}
    \centering
    \includegraphics[width=1\textwidth]{img/cheek_cells}\\
    {\scriptsize Cheeck Cells --- 7th Grade Life Science Classroom Blog}
\end{figure}

Introduz, explica e detalha atividades de entrega e controle para alcançar a
excelência operacional de um serviço. É uma fase mais prolongada do ciclo de
vida, pois o serviço deverá ser mantido em bom estado operacional até que perca
sua utilidade e seja aposentado. O propósito desta fase é coordenar e realizar
as atividades e processos requeridos para gerenciar serviços em níveis
acordados com usuários e clientes.


\section{Conceitos}
\label{sec:opera:conceitos}
\begin{itemize}
    \item Requisição de Serviço: é um pedido por uma mudança ou para acessar um
        serviço de TI.
    \item Evento: notificação criada por um serviço, IC ou ferramenta de
        monitoramento causada pelo desvio de desempenho da infraestrutura ou de
        entrega do serviço.
    \item Alerta: aviso ou advertência sobre uma meta, mudança ou falha que
        ocorreu.
    \item Incidente: interrupção inesperada ou redução na qualidade de um
        serviço de TI.
    \item Problema: causa de um ou mais incidentes.
    \item Solução de Contorno: meio temporário de resolver questões ou
        dificuldades.
    \item Erro Conhecido: problema que tem a causa raíz documentada e uma
        Solução de Contorno identificada.
	\item Base de Erros Conhecidos: local onde se registram Erros Conhecidos.
    \item Impacto, Urgência e Prioridade: É importante avaliar o impacto e a
        urgência de incidentes, problemas ou mudanças nos processos de negócio,
        para determinar suas prioridades. Para o impacto deve-se considerar
        quantas pessoas ou sistemas serão prejudicados. Já a urgência determina
        a velocidade com que o incidente precisa ser resolvido.
\end{itemize}


\section{Princípios}
\label{sec:opera:principios}
\begin{itemize}
    \item Visão Interna do Negócio $\times$ Visão Externa do Negócio: ter
        somente a visão interna, pode levar a focar em sistemas pouco
        importantes ao negócio. Ter só a visão externa, pode levar o pessoal de
        TI a prometer o que não pode cumprir.
    \item Estabilidade $\times$ Agilidade: se a TI pensa apenas na
        estabilidade, ela se torna lenta para se adaptar às necessidades do
        negócio. Se ela é muito ágil, não faz um bom planejamento das mudanças
        e perde estabilidade.
    \item Qualidade do Serviço $\times$ Custo do Serviço: a TI precisa fazer um
        uso ótimo dos recursos.
    \item Reativa $\times$ Pró-ativa: uma TI reativa só age com uma pressão
        externa. Uma TI pró-ativa está sempre buscando oportunidades de
        melhorias ---o que pode levar à perda do foco na necessidade real do
        negócio.
\end{itemize}

A lição que se tira destas comparações é que todos estes aspectos precisam ser
balanceados pelo pessoal de TI. Somente assim será possível prover serviços que
gerem valor ao negócio e ao cliente, de  forma eficiente e eficaz.


\section{Processos}
\label{sec:opera:processos}
\subsection{Gerenciamento de Incidente}
Lida com todos os incidentes ---falhas e dúvidas. A meta deste processo é
restaurar a operação normal do serviço o mais rápido possível e minimizar os
impactos adversos nas operações do negócio. Inclui qualquer evento que
interrompa ou que possa interromper um serviço. Faz uso de alguns conceitos:
\begin{itemize}
	\item Limites de Tempo: acorda os limites de tempo de acordo com os ANSs;
    \item Modelos de Incidente: determinam os passos necessários para executar
        o processo corretamente; e
    \item Incidentes Graves: precisam ser resolvidos com urgência, com a
        utilização de um procedimento específico.
\end{itemize}

Neste processo, além do Gerente de Incidente, há as responsabilidades das
equipes de suporte, que poderão ser agrupadas em vários níveis, cada um mais
especializado que o outro. Assim, caso um nível não consiga diagnosticar o
incidente, ele o repassa para o nível imediatamente acima, mais especializado.


\subsection{Gerenciamento de Evento}
Um evento é qualquer ocorrência detectável ou discernível, que seja
significativa para a gestão da infraestrutura de TI ou para a entrega do
serviço de TI e avaliação do impacto que um desvio pode causar aos serviços.

O Gerenciamento de Evento objetiva proporcionar e fornecer entradas para muitos
processos e atividades da Operação de Serviço. Pode ser aplicado para qualquer
aspecto do Gerenciamento de Serviço que precise ser controlado e que pode ser
automatizado.

Não é necessário possuir o papel de Gerente de Evento, pois suas atividades são
delegadas às funções de TI, como a Central de Serviço.


\subsection{Cumprimento de Requisição}
O termo “cumprimento e requisição” é usado como uma descrição genérica para
muitos tipos de variáveis de demandas colocadas sobre o departamento de TI por
seus usuários. Consiste das seguintes atividades:
\begin{itemize}
	\item Seleção de Menu;
	\item Autorização Financeira;
	\item Cumprimento; e
	\item Conclusão.
\end{itemize}

A propriedade do Cumprimento de Requisição fica com a Central de Serviço, que
monitora, escala, despacha e frequentemente preenche as requisições dos
usuários.


\subsection{Gerenciamento de Problemas}
Tem a intenção de encontrar erros conhecidos na infraestrutura de TI. Os
problemas são a causa de um ou mais incidentes. Um incidente nunca vira
problema: sempre haverá 2 registros separados, um para cada processo.

Neste processo há o envolvimento de 2 papéis: Gerente de Problema e Grupos de
Resolução de Problemas.


\subsection{Gerenciamento de Acesso}
Concede ao usuário o direito de usar um serviço e nega acessos de usuários não
autorizados. É composto de alguns conceitos básicos:
\begin{itemize}
    \item Acesso: é o nível que estende a funcionalidade ou dados que um
        usuário pode usar;
    \item Identidade: informação única que distingue um indivíduo, verifica o
        estado;
	\item Direitos: configurações atuais que permitem o acesso dos usuários;
    \item Serviços dos Grupos de Serviço: é o acesso a um conjunto de serviços
        ou grupos de usuários, mas não o acesso separado aos serviços; e
    \item Serviços de Diretório: refere-se a uma ferramenta que permite
        gerenciar acessos e direitos.
\end{itemize}

Este processo é composto das seguintes atividades: Verificação da Legitimidade
das Requisições, Fornecimento dos Direitos, Monitoramento do Estado de
Identidade ---Mudança de Papéis---, Registro e Monitoramento do Acesso, Remoção
e Limitação de Direitos.

O Gerenciamento de Acesso é uma sobreposição do Gerenciamento de Segurança e do
Gerenciamento de Disponibilidade.


\section{Funções}
\subsection{Central de Serviços}
É uma unidade funcional que está envolvida em vários eventos de serviço. Deve
funcionar como um ponto único de contato para os usuários no dia a dia. Há 4
tipos de Central de Serviços:
\begin{itemize}
    \item Local: é criada para atender necessidades locais de cada unidade de
        negócio;
    \item Centralizada: centraliza todas as solicitações de suporte em um único
        local;
    \item Virtual: suporte espalhado por diversos países – sempre que o usuário
        fizer uma chamada, dependendo do horário, ele será atendido por alguém
        em uma posição geográfica diferente; e
    \item Siga o Sol: é a combinação de centrais que estão dispersas
        geograficamente, fornecendo suporte 24 horas/dia a um custo
        relativamente baixo.
\end{itemize}

A equipe de suporte deverá ter as seguintes qualificações, no mínimo:
Habilidades interpessoais, Entendimento dos serviços e Conhecimento técnico.

Os seguintes papéis figuram nesta função: Gerente de Central de Serviços,
Supervisor de Central de Serviços e Analista de Suporte.


\subsection{Gerenciamento Técnico}
É a função responsável por fornecer habilidades técnicas para o suporte de
serviços de TI e para o Gerenciamento da Infraestrutura de TI. Em pequenas
organizações, pode estar em 1 departamento e em grandes organizações, pode
estar distribuído por vários departamentos.


\subsection{Gerenciamento de Aplicação}
É responsável por gerenciar aplicativos durante seu ciclo de vida. Sua função é
realizada por qualquer departamento, grupo ou equipe envolvida na gestão e
suporte de aplicativos operacionais. Uma das decisões-chave à qual ele
contribui é a de comprar um aplicativo ou criá-lo ---que é discutido no Desenho
de serviço, Capítulo \ref{cha:design}. Esta função também pode ser distribuída
em grupos. O Gerenciamento de Aplicação não é responsável pelo desenvolvimento
do software ---é responsável pela manutenção de aplicações.


\subsection{Gerenciamento de Operações de TI}
É a função responsável pela gestão contínua e manutenção de uma infraestrutura
de TI de uma organização, para assegurar a entrega do ANS ao negócio. Consiste
de 2 sub-funções: Controle de Operações de TI e Gerenciamento das Instalações.
