% Hevea - ITILv3 Resumo
% An ITILv3 overview.
%
% Author: José Lopes de Oliveira Jr. <jilo.cc>
%
% LICENSE
% This program is free software: you can redistribute it and/or modify
% it under the terms of the GNU General Public License as published by
% the Free Software Foundation, either version 3 of the License, or
% (at your option) any later version.
%
% This program is distributed in the hope that it will be useful,
% but WITHOUT ANY WARRANTY; without even the implied warranty of
% MERCHANTABILITY or FITNESS FOR A PARTICULAR PURPOSE.  See the
% GNU General Public License for more details.
%
% You should have received a copy of the GNU General Public License
% along with this program.  If not, see <http://www.gnu.org/licenses/>.
%%


\chapter{Introdução}
\label{cha:intro}


\begin{figure}
    \centering
    \includegraphics[width=1\textwidth]{img/bacteria_bunch}\\
    {\scriptsize Bunch of Bacteria --- Mirror Lake}
\end{figure}

Ao longo do tempo, a Tecnologia da Informação (TI) passou a obter mais destaque
dentro das organizações. Com isso, o setor passou a ser menos tolerante a
falhas e criou-se a ideia de que ele deveria deixar o posto de simples provedor
de tecnologia, para se tornar um verdadeiro parceiro de negócio para a
organização.
\nomenclature{TI}{Tecnologia da Informação}%

Neste contexto surgiram frameworks de processos e boas práticas. Tais
frameworks buscavam guiar a TI de forma a se organizar melhor e alcançar seu
objetivo. Surgiram também normas para empresas com setores de TI validados
dentro de um determinado padrão, como a ISO 20000.
\nomenclature{ISO}{International Organization for Standardization}%

\section{ITIL\textsuperscript{\textregistered}}
\label{sec:intro:itil}

A ISO 20000 certifica empresas, mas não pessoas. Enquanto norma de
certificação, ela aponta o que se espera que as empresas façam, mas não mostra
em detalhes como fazer isso. A Information Technology Infrastructure Library
(ITIL\textsuperscript{\textregistered}) surge como mais uma solução para
preencher essas lacunas.
\nomenclature{ITIL}{Information Technology Infrastructure Library}%

Desenvolvida pelo Central Computing and Telecommunications Agency\\ (CCTA),
atual Office of Government Commerce (OGC), a
ITIL\textsuperscript{\textregistered} surgiu em 1980 e tornou-se padrão
\emph{de facto} do mercado 10 anos depois. Em 2000, sofreu sua primeira grande
revisão, dando origem à ITIL\textsuperscript{\textregistered}v2 e 7 anos mais
tarde, foi concluída outra revisão que deu origem à
ITIL\textsuperscript{\textregistered}v3.
\nomenclature{CCTA}{Central Computing and Telecommunications Agency}%
\nomenclature{OGC}{Office of Government Commerce}%

Apesar de ser desenvolvida pelo OGC, que é um departamento do governo
britânico, há outras entidades envolvidas com a
ITIL\textsuperscript{\textregistered}. Como o governo britânico não tem
interesse em ganhar dinheiro com este material, ele se limita a gerenciar as
atualizações principais e imprimir os livros. Já que o mercado passou a
considerar importante a certificação de profissionais neste assunto, o OGC
delegou esta tarefa a órgãos não governamentais.

Atualmente, o responsável direto pela certificação de profissionais
ITIL\textsuperscript{\textregistered} é o APM Group (APMG). Mas este órgão
apenas gerencia o esquema de certificação e distribui produtos
ITIL\textsuperscript{\textregistered}. Para desenvolverem uma certificação
profissional para ITIL\textsuperscript{\textregistered}, foram destacados o
Examination Institute for Information Science (EXIN) e o Information Systems
Examinations Board (ISEB).
\nomenclature{APMG}{APM Group Ltd}%
\nomenclature{EXIN}{Examination Institute for Information Service}%
\nomenclature{ISEB}{Information Systems Examinations Board}%

Dessa forma, para se certificar em ITIL\textsuperscript{\textregistered}, o
profissional deve procurar um centro de exames credenciado pelo EXIN ou ISEB.
Ele fará o seu cadastro, marcará e realizará sua prova de certificação.

Cabe ressaltar que existe o Information Technology Service Management Forum
(itSMF), que agrega membros ---empresas e profissionais---, fornecendo um meio
para troca de informações e experiências entre os associados. Normalmente, cada
país tem sua divisão ---capítulo--- do itSMF.
\nomenclature{itSMF}{Information Technology Service Management Forum}%

\section{Certificação}
\label{sec:intro:certi}

A ITIL\textsuperscript{\textregistered}v3 é organizada em 5 livros. O conteúdo
superficial destes livros é cobrado no nível de certificação mais básico:
Foundation. Para conseguir esta certificação, o profissional não precisa
participar de cursos oficiais, nem ter formação na área de TI. Basta estudar o
conteúdo proposto e obter 65\% de acerto na prova de 40 questões ---que deve
ser realizada em até 1 hora, em um dos centros credenciados.

Ao conseguir a aprovação, o profissional receberá um broche, juntamente com o
seu certificado. Este material é enviado diretamente pelo órgão de certificação
a que ele se submeteu ---EXIN ou ISEB. A prova custa em torno de US\$ 150,00 e
é vitalícia. Isso quer dizer que quem se certificou em
ITIL\textsuperscript{\textregistered}v2, não precisa fazer a prova para
ITIL\textsuperscript{\textregistered}v3, pois será um profissional
ITIL\textsuperscript{\textregistered}v2 para sempre.

O nível máximo que se pode chegar em ITIL\textsuperscript{\textregistered}v3 é
o Expert e para isso, precisa-se obter 22 pontos em cursos e provas oficiais.
Para se ter uma ideia, ao obter o ITIL\textsuperscript{\textregistered}
Foundation, o profissional ``ganha'' 2 pontos.

As provas para o ITIL\textsuperscript{\textregistered}v3 Foundation já estão
disponíveis em Português do Brasil e a Prometric é uma das agências mais
flexíveis para se obter essa certificação ---pelo EXIN.

\section{Livros}
\label{sec:intro:livros}

Os 5 livros-base da ITIL\textsuperscript{\textregistered}v3 demostram cada
etapa do ciclo de vida do serviço. Dessa forma, cada livro possui um conjunto
de processos, papéis e funções para realização de cada etapa. Vale lembrar que
a ITIL\textsuperscript{\textregistered} deve ser adequada à realidade de cada
organização. Ela dá ideias do \textbf{quê} pode ser feito, mas não de
\textbf{como} fazer. De fato, a ITIL\textsuperscript{\textregistered} é
bastante flexível e isso se reflete na sua constituição: ela é organizada na
forma de processos, o que a torna aplicável em praticamente qualquer estrutura
de TI. Estes são os os assuntos abordados nos livros da
ITIL\textsuperscript{\textregistered}:
\begin{itemize}
    \item Estratégia de Serviço: descreve como a TI vai se integrar ao negócio,
        deixando de ser reativa e passando a ser proativa.
    \item Desenho de Serviço: projeta o serviço descrito na Estratégia de
        Serviço e define o Acordo de Nível de Serviço (ANS) com o cliente.
    \item Transição de Serviço: coloca o serviço em operação, preparando o
        ambiente para isso.
    \item Operação de Serviço: mantém o serviço funcionando no cotidiano.
    \item Melhoria de Serviço Continuada: avalia o serviço e os processos de
        gerenciamento das fases. Busca melhorar a qualidade, sabendo que um
        serviço não é estável.
\end{itemize}
