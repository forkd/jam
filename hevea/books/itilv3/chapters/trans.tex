% Hevea - ITILv3 Resumo
% An ITILv3 overview.
%
% Author: José Lopes de Oliveira Jr. <jilo.cc>
%
% LICENSE
% This program is free software: you can redistribute it and/or modify
% it under the terms of the GNU General Public License as published by
% the Free Software Foundation, either version 3 of the License, or
% (at your option) any later version.
%
% This program is distributed in the hope that it will be useful,
% but WITHOUT ANY WARRANTY; without even the implied warranty of
% MERCHANTABILITY or FITNESS FOR A PARTICULAR PURPOSE.  See the
% GNU General Public License for more details.
%
% You should have received a copy of the GNU General Public License
% along with this program.  If not, see <http://www.gnu.org/licenses/>.
%%


\chapter{Transição de Serviço}
\label{cha:trans}


\begin{figure}
    \centering
    \includegraphics[width=0.7\textwidth]{img/ant_microscope}\\
    {\scriptsize An Army of One --- Arthtopod Image Salon}
\end{figure}

Esta fase ajuda a organização a planejar mudanças nos serviços e implantar
liberações de serviço com sucesso no ambiente de produção. Assegura que haja o
mínimo de impacto no serviços em produção quando uma mudança ou novo serviço
for implantado. Além disso, garante que os requisitos da Estratégia do Serviço
estejam definidos no Pacote de Desenho de Serviço.


\section{Conceitos}
\begin{itemize}
    \item Item de Configuração (IC): é um ativo, componente do serviço ou
        qualquer outro item que está ou estará sob o controle do processo de
        Gerenciamento da Configuração. Exemplos: servidores, aplicativos,
        documentações de processos, Pacote de Desenho de Serviço, Plano de
        Negócio.
    \item Banco de Dados de Gerenciamento da Configuração (BDGC): é um
        repositório de informações onde serão armazenados os registros de ICs.
        Cada IC armazenado no BDGC deve ter um identificador único. Usando
        várias aplicações para esta finalidade, a organização poderá ter vários
        BDGCs.
    \item Sistema de Gerenciamento de Configuração (SGC): armazena todas as
        informações dos ICs dentro de um escopo determinado. Os diversos BDGCs
        que existirem na organização precisam ser integrados para que as
        informações não fiquem duplicadas e desatualizadas.
    \item Sistema de Gerenciamento de Conhecimento de Serviço (SGCS): é formado
        por um conjunto de dados em base central. Os BDGCs alimentam o SGC, que
        fornece informações para o SGCS e elas suportam os processos de tomada
        de decisão. A forma com que as pessoas usam as informações gera o
        conhecimento. Não há como armazenar conhecimento, pois ele está
        relacionado a experiências e habilidades das pessoas.
    \item Biblioteca de Mídia Definitiva (BMD): é uma biblioteca segura na qual
        versões autorizadas definitivas de todas as mídias de ICs são
        armazenadas e protegidas ---está mais relacionada aos meios físicos de
        armazenamento do que o BDGC.
    \item Mudança de Serviço: designa uma mudança em um serviço existente ou
        uma introdução de novo serviço no ambiente de produção.
	\item Tipos de Mudanças:
        \begin{itemize}
            \item Padrão: mudança em um serviço ou infraestrutura que é
                pré-autorizada pelo Gerenciamento de Mudança ---é uma rotina
                definida por um script padrão.
            \item Normal: é levantada por um iniciador que requer uma mudança
                ---precisa ser autorizada antes de executada.
            \item Emergencial: precisa ser implantada rapidamente para resolver
                um incidente, por isso nem sempre pode-se realizar todos os
                testes.
        \end{itemize}

	\item 7 Rs do Gerenciamento de Mudança:
        \begin{enumerate}
            \item Raise (Quem submeteu a mudança?)
            \item Reason (Qual é a razão da mudança?)
            \item Return (Qual é o retorno requerido a partir da mudança?)
            \item Risks (Quais são os riscos envolvidos na mudança?)
            \item Resources (Quais são os recursos necessários para entregar a
                mudança?)
            \item Responsible (Quem é o responsável pela mudança?)
            \item Relationship (Qual é a relação entre esta mudança e outras
                mudanças?).
        \end{enumerate}

    \item Unidade de Liberação: descreve a porção de um serviço ou
        infraestrutura de TI que é normalmente liberada de acordo com a
        política de liberação da organização.
    \item Modelo V de Serviço: serve como uma ferramenta para mapear os
        diferentes níveis de configuração que precisam ser construídos e
        testados. Quanto mais cedo os testes forem feitos, mais cedo as falhas
        serão descobertas, evitando-se o retrabalho.
\end{itemize}
\nomenclature{IC}{Item de Configuração}
\nomenclature{BDGC}{Banco de Dados de Gerenciamento da Configuração}
\nomenclature{SGC}{Sistema de Gerenciamento da Configuração}
\nomenclature{SGCS}{Sistema de Gerenciamento de Conhecimento de Serviço}
\nomenclature{BMD}{Biblioteca de Mídia Definitiva}


\section{Processos}
\label{sec:trans:processos}
\subsection{Gerenciamento de Mudança}
Assegura que mudanças são feitas de forma controlada e são avaliadas,
priorizadas, planejadas, testadas, implantadas e documentadas. Gerenciar
mudanças não é fazer mudanças que não ofereçam riscos, mas tornar os riscos
mapeados e gerenciados. Se a TI for muito ágil para implantar as mudanças, pode
provocar incidentes, interrupções e retrabalhos. Se a TI for demasiadamente
burocrática, pode prejudicar o negócio do cliente. O escopo do Gerenciamento de
Mudança cobre as mudanças desde a base de ativos de serviço e ICs até o
completo ciclo de vida do serviço. Cada organização deve definir as mudanças
que ficarão fora do escopo do seu processo de Gerenciamento de Mudança.

Requisição de Mudança (RDM) é uma requisição formal para mudar um ou mais ICs.
O Comitê Consultivo de Mudanças (CCM) é formado por pessoas que se reúnem para
autorizar a mudança e assistir na sua avaliação e priorização.
\nomenclature{RDM}{Requisição de Mudança}
\nomenclature{CCM}{Comitê Consultivo de Mudanças}

O primeiro passo é registrar uma RDM, então os \emph{stakeholders} revisam a
RDM, verificando se ela está em conformidade. Em seguida, a mudança é analisada
e avaliada ---entram aqui os 7 Rs--- e o CCM ou o Gerente de Mudança determina
se a mudança será implantada. Caso seja, determina-se sua prioridade com base
no impacto e na urgência. A próxima etapa é a coordenação da implantação, onde
o Gerenciamento de Liberação e Implantação coordenará as atividades envolvidas
na construção e criação de liberações. Uma vez implantada, a mudança passará
por uma avaliação para checar se ela cumpriu o seu propósito.

O Gerente de Mudança será o responsável por este processo, atuando desde o
registro da RDM junto ao cliente, até a revisão da mudança e a produção de
relatórios.


\subsection{Gerenciamento da Configuração e de Ativo de Serviço}
É o processo que identifica todos os ICs necessários para entregar os serviços
de TI. Certifica que as informações sobre os ICs nos BDGCs são corretas e
atuais. Alguns conceitos são definidos aqui.
\begin{itemize}
	\item Biblioteca Segura: coleção de ICs.
	\item Armazém Seguro: local onde podem ser armazenados ativos de TI.
    \item Biblioteca de Mídia Definitiva: é a Biblioteca Segura onde versões de
        software autorizadas são armazenadas.
    \item Peças Definitivas: Armazém Seguro onde estão as peças sobressalentes
        de hardware.
    \item Linha de Base de Configuração: configuração aprovada de um serviço,
        produto ou infraestrutura.
	\item Instantâneo (Snapshot): cópia do estado atual de um IC ou ambiente.
\end{itemize}

Vários papéis são definidos aqui. Gerente de Ativos de Serviço, Gerente de
Configuração, Analista de Configuração, Bibliotecário de Configuração,
Administrador de Ferramentas SGC.


\subsection{Gerenciamento de Liberação e Implantação}
Faz o controle de versões e controla as instalações de software, hardware e
outros componentes de infraestrutura, do ambiente de desenvolvimento ao
ambiente de testes e depois para o ambiente de produção.Vai assegurar que haja
o mínimo impacto não precedente nos serviços de produção e na organização de
operações e suporte.

Unidade de Liberação é a parte do serviço ou infraestrutura que está incluída
na liberação de acordo com as diretrizes de liberação da organização. A forma
de distrubuição pode ser:
\begin{itemize}
    \item Big Bang: implanta o serviço novo ou alterado para todos os usuários
        ao mesmo tempo;
	\item Fase: é feita para parte dos usuários;
    \item Empurrada: o componente de serviço é implantado à partir da área
        central para usuários em locais remotos;
	\item Puxada: o sistema do usuário é responsável por buscar a atualização; e
    \item Automatizada ou Manual: liberações maiores podem ser automatizadas e
        menores podem ser feitas manualmente.
\end{itemize}

Um Pacote de Liberação pode ser a única Unidade de Liberação ou uma coleção
delas. Este processo consiste basicamente das seguintes atividades:
\begin{itemize}
	\item Planejamento
	\item Preparação para Construção
	\item Teste e Implantação
	\item Construção e Teste
	\item Teste de Serviço e Pilotos
	\item Planejamento para Implantação
	\item Transferência
	\item Implantação e Retirada
	\item Verificação da Implantação
	\item Suporte para o Período de Funcionamento Experimental
\end{itemize}

Os papéis definidos aqui são: Gerente de Liberação e Implantação, Gerente de
Empacotamento e Construção de Liberação, Equipe de Implantação.
