% Hevea - ITILv3 Resumo
% An ITILv3 overview.
%
% Author: José Lopes de Oliveira Jr. <jilo.cc>
%
% LICENSE
% This program is free software: you can redistribute it and/or modify
% it under the terms of the GNU General Public License as published by
% the Free Software Foundation, either version 3 of the License, or
% (at your option) any later version.
%
% This program is distributed in the hope that it will be useful,
% but WITHOUT ANY WARRANTY; without even the implied warranty of
% MERCHANTABILITY or FITNESS FOR A PARTICULAR PURPOSE.  See the
% GNU General Public License for more details.
%
% You should have received a copy of the GNU General Public License
% along with this program.  If not, see <http://www.gnu.org/licenses/>.
%%


\chapter{Fundamentos}
\label{cha:fund}


\begin{figure}
    \centering
    \includegraphics[width=1\textwidth]{img/fish_xray}\\
    {\scriptsize Fish X-Ray --- Arstochromis Christyi}
\end{figure}

Serviço é um meio de entregar valor aos clientes, facilitando os resultados que
eles querem alcançar, sem ter que assumir custos e riscos. Quando se adquire um
carro, por exemplo, o comprador quer que o mesmo se locomova sem ter que se
preocupar com os componentes mecânicos que possibilitam isso.

O Gerenciamento de Serviços de TI (Information Technology Service Management
--- ITSM) é um conjunto de habilidades da organização para fornecer valor ao
cliente em forma de serviços. Quando uma organização passa a depender da TI
para funcionar, esta pode ser encarada como um Ativo Estratégico.
\nomenclature{ITSM}{Information Technology Service Management}%

Ativos de Serviços ---Recursos e Habilidades--- são usados pelo provedor de TI
para entregar um serviço. Recursos são necessários para a produção de um bem ou
realização de um serviço ---e.g., um computador, software, capital financeiro.
Habilidades são a capacidade  da organização de usar os recursos para produzir
valor ---e.g., conhecimento, processos, organização.

Para se criar valor para os serviços é necessário unir Utilidade e Garantia.
Utilidade diz respeito ao que o cliente quer ---e.g., imprimir um documento.
Garantia diz respeito a como o cliente quer receber o serviço ---e.g., como a
impressora é gerenciada para imprimir documentos.


\section{Processos, Funções e Papéis}
\label{sec:funda:proc}

Estes são termos importantíssimos dentro da
ITIL\textsuperscript{\textregistered}, sendo o seu entendimento primordial para
o sucesso da certificação.

Uma organização normalmente é composta por setores. Isso é ruim, pois dificulta
a comunicação e tira o foco dos funcionários ---em vez de pensarem no cliente,
pensam no seu gerente. Processos buscam resolver esses problemas.

\textbf{Processo} é um conjunto estruturado de atividades para produzir um
determinado resultado. Processos não podem existir sozinhos. Alguém precisa
executá-los. Então existem as funções.

\textbf{Função} é um time ou grupo de pessoas e ferramentas usadas para um ou
mais processos e atividades. Como é necessário que todo processo tenha um
proprietário e responsáveis por executá-lo, surgem os papéis.

\textbf{Papel} é um conjunto de responsabilidades e autoridades concedidas a
uma pessoa ou time.

É importante notar que  um papel não é um cargo, assim como um processo não é
um setor dentro de uma empresa. Dessa maneira, uma pessoa pode desempenhar
vários papéis e a implantação dos processos torna-se independente da estrutura
organizacional da empresa.

A ITIL\textsuperscript{\textregistered}v3 é composta por 5 etapas que juntas
têm um total de 24 processos. Cada processo é iniciado por um gatilho e passa
ser controlado e desenvolvido pelas pessoas nas funções específicas dele. Essas
pessoas usam habilidades e recursos para realizar cada atividade que compõe o
processo, gerando a saída do mesmo.

Note que um processo deve ser mensurável e orientado ao cliente. Além disso,
todo processo tem um proprietário, que assegura que o desenvolvimento
transcorra como acordado e documenta tudo isso, para atingir os objetivos
propostos. Já o Proprietário do Serviço, faz o primeiro contato com o cliente e
é responsável pelo serviço como um todo.


\section{Matriz RACI}
\label{sec:funda:raci}

A matriz RACI é usada para definir papéis para execução de um processo. Ela
recebe este nome por ser composta por 4 especificações:
\nomenclature{RACI}{Responsible, Accountable, Consulted, Informed}%
\begin{itemize}
    \item Responsible: os responsáveis pela atividade.
    \item Accountable: prestador de conta da atividade (normalmente é o
        proprietário do processo --- somente uma pessoa pode receber esta
        atribuição dentro de uma atividade).
    \item Consulted: pessoas que serão consultadas na necessidade de
        compartilhamento de uma informação.
    \item Informed: pessoas que devem ser informados durante o progresso.
\end{itemize}


\section{Ciclo de Vida do Serviço}
\label{sec:funda:ciclo}

É composto por fases que levam o nome e são definidas em cada um dos 5 livros
da ITIL\textsuperscript{\textregistered}v3: Estratégia de Serviço, Desenho de
Serviço, Transição de Serviço, Operação de Serviço e Melhoria de Serviço
Continuada.

Cada fase do ciclo de vida recebe como entrada, a saída da anterior e gera
entradas para a próxima. É interessante notar que a Melhoria de Serviço
Continuada, apesar de aparecer por último na sequência, faz parte de todo o
ciclo de vida. Dessa forma, não se chega num ponto para buscar a melhoria do
serviço: ela deve estar presente em todas as fases.
