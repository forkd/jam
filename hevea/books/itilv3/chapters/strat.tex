% Hevea - ITILv3 Resumo
% An ITILv3 overview.
%
% Author: José Lopes de Oliveira Jr. <jilo.cc>
%
% LICENSE
% This program is free software: you can redistribute it and/or modify
% it under the terms of the GNU General Public License as published by
% the Free Software Foundation, either version 3 of the License, or
% (at your option) any later version.
%
% This program is distributed in the hope that it will be useful,
% but WITHOUT ANY WARRANTY; without even the implied warranty of
% MERCHANTABILITY or FITNESS FOR A PARTICULAR PURPOSE.  See the
% GNU General Public License for more details.
%
% You should have received a copy of the GNU General Public License
% along with this program.  If not, see <http://www.gnu.org/licenses/>.
%%


\chapter{Estratégia de Serviço}
\label{cha:estrat}


\begin{figure}
    \centering
    \includegraphics[width=1\textwidth]{img/cell_biology}\\
    {\scriptsize Cell Biology --- The University of New Mexico}
\end{figure}

Nesta fase, todo o planejamento com relação ao serviço será feito, de forma com
que a TI se alinhe ao negócio, tornando-se um ativo estratégico. É preciso
entender que todo serviço de TI tem como propósito sustentar um processo de
negócio.

A fase de Estratégia de Serviço usa uma abordagem multidisciplinar para o
desenvolvimento de serviços. Dessa forma, profissionais de várias áreas dentro
da empresa são envolvidos. O objetivo disso vem da seguinte reflexão: ``O
cliente não compra um serviço ou produto. Ele compra uma solução para resolver
necessidades específicas.''

A TI precisa se alinhar ao negócio para trabalhar de acordo com as necessidades
dos clientes e \emph{stakeholders} (envolvidos no negócio).


\section{4 Ps da Estratégia}
\label{sec:estrat:4ps}

É um conceito desenvolvido no livro Estratégia do Serviço (Mintzberg).
\begin{itemize}
    \item Perspectiva: dá a direção para o provedor de serviço ---missão, visão
        e valores da organização.
	\item Posição: é a imagem que a organização quer passar para os clientes.
	\item Plano: descreve como a organização vai executar a estratégia.
    \item Padrão: é resultado dos 3 Ps anteriores e representa os procedimentos
        da organização, as ações que ela vai tomar.
\end{itemize}


\section{Criação de Valor}
Acontece quando se une os efeitos da Utilidade e da Garantia. É preciso que a
TI crie valor para a organização, para justificar sua existência.


\section{Portfolio de Serviços}
É a representação de todos os serviços de TI e seus estados. É composto por 3
componentes:
\begin{itemize}
    \item Pipeline de Serviços: também conhecido como Funil de Serviços, irá
        conter todos os serviços futuros ---propostos ou em desenvolvimento. É
        a famosa lista de pedidos de aplicações e funcionalidades que os
        usuários enviam. Com o alinhamento da TI ao negócio, quem decide o que
        será implementado é a estratégia da organização.
    \item Catálogo de Serviços: contém os serviços de TI que são oferecidos aos
        clientes e aqueles que foram liberados e entrarão em operação em breve.
    \item Serviços Obsoletos: serviços que estavam em operação e já foram
        aposentados.
\end{itemize}


\section{Catálogo de Serviço}
É como um menu de restaurante e define os serviços acessíveis pelos clientes de
forma mais descritiva que o Portfolio de Serviços. Há 2 tipos:
\begin{itemize}
    \item Catálogo de Serviço do Negócio: está disponível para os clientes, em
        uma linguagem apropriada a eles.
    \item Catálogo de Serviço Técnico: não está visível para o cliente, pois
        contém detalhes técnicos de todos os serviços entregues.
\end{itemize}


\section{Business Case (Plano de Negócio)}
É um documento que possui todos os requisitos de qualidade e custos para que um
serviço opere.


\section{Riscos}
São definidos como um resultado incerto, uma oportunidade positiva ou uma
ameaça. Para gerenciá-los, deve-se fazer um controle das vulnerabilidades,
usando algum framework. Tem 2 etapas distintas:
\begin{itemize}
	\item Análise de Risco: coleta informações sobre a exposição ao risco.
    \item Gerenciamento dos Riscos: gera processos para monitorar e acessar
        informações sobre os riscos, além de processos de tomada de decisão.
\end{itemize}


\section{Modelos de Serviço}
São a forma com que o provedor pretende entregar o serviço ao cliente.


\section{Atividades}
Colocam em prática a estratégia da organização. São elas:
\begin{itemize}
    \item Definir o mercado: entender o cliente e as oportunidades. Classificar
        e visualisar os serviços.
    \item Desenvolver a oferta: definição dos serviços baseados em resultados.
        Portfolio, funil e catálogo de serviço. Utilidade + Garantia: como o
        serviço vai gerar valor para o cliente?
    \item Desenvolver ativos estratégicos: gerenciar o serviço como um ativo
        estratégico para a organização.
    \item Preparar para execução: fazer análise do serviço ---estratégia,
        fatores críticos, competição--- e priorizar investimentos. Matriz SWOT
        ---pontos fortes, fracos, oportunidades e ameaças.
\end{itemize}
\nomenclature{SWOT}{Strengths, Weaknesses, Opportunities, Threats}


\section{Processos}
\subsection{Gerenciamento Financeiro}
As organizações de TI também têm a necessidade de analisar, empacotar, vender e
entregar serviços assim como qualquer empresa. Este processo ajuda na tomada de
decisões ---é preciso saber quanto custa para desenvolver e manter um novo
serviço.

Ajuda também a saber se o serviço gera, de fato, valor para o negócio. É
composto por Orçamento, Contabilidade de TI e Cobrança ---este último, quando
aplicável.

TI é um provedor de serviços e não um provedor de tecnologia.

O Gerente Financeiro será o papel do responsável por este processo.


\subsection{Gerenciamento da Demanda}
Excesso ou falta de capacidade podem gerar perdas para o negócio. Aqui podem
ser definidos ANSs, previsões, planejamento e coordenação com o cliente, para
reduzir a incerteza da demanda ---mas não eliminá-la, pois isso é impossível.
\nomenclature{ANS}{Acordo de Nível de Serviço}

Este processo analisa, rastreia, monitora e documenta os Padrões de Atividade
do Negócio (PAN), que dirão como o cliente usa os serviços e quais os períodos
de pico na utilização deles.
\nomenclature{PAN}{Padrões de Atividade do Negócio}%

O Gerente de Demanda ---papel--- será o responsável por este processo, que,
dentre outras coisas, participará na criação dos ANSs, responderá às mudanças
no PAN e fará o monitoramento da demanda e da capacidade.


\subsection{Gerenciamento do Portfolio de Serviço}
Apenas gerencia os serviços e seus estados, não entrando em detalhes na
documentação de funcionalidades do serviço dentro do Catálogo de Serviço.

Define o portfolio, criando um inventário de serviços e validando os dados.
Analisa o portfolio, definindo quais serviços servem apenas para o negócio que
quais o farão crescer. Aprova o portfolio, autorizando serviços e recursos para
o futuro e eliminando serviços do portfolio atual. Contrata serviços,
comunicando decisões, alocando recursos e fornecendo todo o planejamento para
começar a próxima etapa do ciclo de vida.

O papel criado neste processo é o de Gerente de Produto, que é uma função comum
na área de marketing. Ele é quem vai gerenciar os serviços como produtos e
avaliará as novas oportunidades de mercado, modelos de operação, tecnologia e
necessidades dos clientes.
