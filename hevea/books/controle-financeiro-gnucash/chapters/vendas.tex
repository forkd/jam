ass="alignright size-medium wp-image-168" title="Pirate Gold - Bakati" src="http://joselop.es/app/wor/wp-content/uploads/2010/12/pirate_gold-300x225.png" alt="Pirate Gold - Bakati" width="300" height="225" />Para vender à vista, abra a conta onde o bem está declarado no GNUCash e faça uma transação no valor de venda do mesmo para a conta de destino.

Considere a venda de um automóvel que está cadastrado em Ativos:Ativos Fixos:Automóvel. Considere ainda que o dinheiro da venda tenha ido para a sua conta corrente. Para cadastrar isso no GNUCash, bastaria fazer uma transação no valor da venda entre Ativos:Ativos Fixos:Automóvel e Ativos:Ativos Atuais:Conta Corrente.

O problema disso é que a conta onde o automóvel foi declarado tem de ser zerada. Se você vender o bem por um preço maior que o de compra, significa que houve valorização do mesmo. De forma contrária, houve desvalorização.
<h3>Valorização</h3>
Para lidar com a venda de um bem valorizado, faça a transferência do capital total de compra do bem para a conta onde o dinheiro foi. Depois pegue a diferença entre o preço de compra e o de venda e lance como uma transação à partir da conta Receitas:Outras Receitas, por exemplo.

No caso da venda do carro por R$ 30.000,00, considere que ele tenha sido comprado por R$ 25.000,00. Para fazer esse lançamento, crie uma transação entre Ativos:Ativos Fixos:Automóvel e Ativos:Ativos Atuais:Conta Corrente no valor de R$ 25.000,00, descrevendo a mesma como Venda Carro. Lance os R$ 5.000,00 restantes como uma transação entre Receitas:Outras Receitas e Ativos:Ativos Atuais:Conta Corrente. No campo Descrição, coloque a mesma indicada na primeira transação, adicionando a palavra Valorização ao final.
<h3>Desvalorização</h3>
Se você entendeu como fazer o lançamento da valorização do bem, fica fácil explicar a desvalorização. Neste caso, a diferença entre o preço de compra e o de venda deve ser lançado como uma despesa. No exemplo do carro, suponha que ele tenha sido comprado por R$ 40.000,00 e está sendo vendido por R$ 30.000,00.

Faça uma transação no valor de R$ 30.000,00 entre Ativos:Ativos Fixos:Automóvel e Ativos:Ativos Atuais:Conta Corrente. Para os R$ 10.000,00 restantes, faça uma transação para Despesas:Automóvel, indicando a venda do carro na descrição, com a palavra Desvalorização (ou Depreciação) ao final.
<h3>A Prazo</h3>
- Vou demorar a ver o dinheiro, mas vou receber... espero...
Não venda a prazo sem ter garantias de recebimento do dinheiro, OK?

Crie a conta Ativos:Contas a Receber - como mostrado no texto sobre Empréstimos. Faça uma transação no valor da venda, entre a conta onde o bem foi declarado e a conta recém criada. A cada parcela recebida, transfira o valor da mesma para a conta de destino do dinheiro.

A dica que eu dou aqui é usar o mesmo esquema de pagamento parcelado no cartão de crédito: adicione a discriminação do número da parcela paga/total de parcelas ao final da descrição (e.g., Pagamento da parcela do carro 43/48).
<h3>Valorização e Desvalorização</h3>
Para essas complicações, você pode usar os mesmos esquemas ensinados na venda à vista. ;-)

Bem, esse é o fim da série que eu havia programado. Creio que apresentei muitas coisas que podem ser feitas com o GNUCash, possibilitando ao leitor uma melhor experiência com o programa. Espero que aqueles que já leram tenham gostado!

- Ah não... =(

Obrigado pelas palavras de apoio ao longo desses meses e que as dicas passadas sirvam para melhorar a vida de todos! Lembre-se sempre: é imprecindível usar o dinheiro com responsabilidade. Mas não é o dinheiro que nos faz. Ao contrário: nós fazemos o dinheiro! Não seja escravo dele! Faça-o trabalhar pra você e trate-o com respeito.
