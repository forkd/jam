ass="alignright size-medium wp-image-165" title="Cash Flow - Gregg Kerr Action Coach" src="http://joselop.es/app/wor/wp-content/uploads/2010/12/cash_flow-300x199.jpg" alt="Cash Flow - Gregg Kerr Action Coach" width="300" height="199" />Para lançar o seu salário, abra a conta Receitas:Salário - expanda a conta Receitas e dê um duplo clique sobre a subconta Salário para abri-la em uma nova aba.

Note que nessa nova aba, os nomes das colunas mudaram. Note também que há um lançamento feito referente ao saldo inicial (aquele que foi feito quando da primeira utilização do programa). Há uma linha em branco após esse lançamento inicial: é aí que faremos o lançamento do salário recebido. Acompanhe, coluna por coluna:
<ol>
    <li>Data. Pode deixar a indicação da data atual ou altere em casos especiais;</li>
    <li>Número. É um identificador da transação. Pode ser deixado em branco, mas eu gosto de preenchê-lo de acordo com a data e a hora, gerando um número identificador único, e.g., 200901191141 (Ano, Mês, Dia, Hora, Minuto);</li>
    <li>Descrição. A descrição da operação. Gosto de utilizar palavras chave, para saber de cara o que aquela transação significa, e.g. <strong>Pagamento</strong> da Internet, <strong>Compra</strong> do carro, <strong>Venda</strong> da casa, <strong>Recebimento</strong> de juros. Coloque algo como Recebimento de salário de Tal Empresa.</li>
    <li>Transferência. Indica para onde o dinheiro vai. No meu caso, quando eu recebo, meu dinheiro vai para a conta corrente, então eu indico Ativos:Ativos Atuais:Conta Corrente (felizmente o GNUCash é muito esperto e facilita este preenchimento através de memorização de itens já escritos e lista de itens).</li>
    <li>Receita. Preencha esta coluna com o valor recebido e tecle Enter para finalizar a transação.</li>
</ol>
Feito isso, abra a conta para onde enviou o dinheiro e observe que o dim-dim tá lá! Agora lembre-se de usar este dinheiro com responsabilidade e inteligência. Poupe sempre que puder!
<h3>Pagando uma Conta</h3>
- Temos mesmo que falar nisso?

Temos sim! É importante manter as contas em dia para evitar pagar juros - além de ficar mal visto pelos outros. Considere que você paga R$ 90,00 de Internet todo mês e que agora tem que fazer o lançamento deste pagamento. Suponha ainda que você realize este pagamento da sua conta corrente, através do Internet Banking.
<ol>
    <li>Abra a sua conta corrente em uma nova aba.</li>
    <li>Na coluna Número, coloque o identificador da transação (opcional).</li>
    <li>Em Descrição, coloque algo como Pagamento da Internet.</li>
    <li>Em Transferência, indique Despesas:Internet.</li>
    <li>Em Saque, coloque 90 e tecle Enter.</li>
    <li>Feito isso, serão debitados R$ 90,00 da sua conta corrente. Este dinheiro será enviado para Despesas:Internet, compondo o histórico de tudo o que você gastou com Internet.</li>
</ol>

<h3>Poupança</h3>
- Sempre que sobra alguma coisa no final do mês, eu guardo na minha conta poupança!

Parabéns! É sempre bom guardar dinheiro. Não temos uma bola de cristal para prever o que nos aguarda no futuro. Assim, é sempre bom estar prevenido. De qualquer forma, o ideal é fixar uma porcentagem dos seus ganhos para guardar. Seja ela 10%, 25% ou 50%, fixe bem esse valor e comprometa-se em guardá-lo mensalmente.

Evite também desculpas para si mesmo que o levem a quebrar o seu planejamento. Se você ganha R$ 1.000,00 e comprometeu-se consigo mesmo a guardar 10% do seu salário, aprenda a viver com R$ 900,00 por mês ou arrume outra fonte de renda. Os R$ 100,00 do seu investimento têm de ser sagrados.

Vejamos como fazer esse lançamento no GNUCash. Imagine que o seu salário de R$ 1.000,00 chegou e você já o lançou no programa, como descrito anteriormente. Agora você quer representar no programa a transferência que acabou de fazer de R$ 100,00 para a sua conta poupança. Siga os passos:
<ol>
    <li>considerando que o seu dinheiro esteja na Conta Corrente, acesse e expanda essa conta;</li>
    <li>crie uma nova transação para Ativos:Ativos Atuais:Conta Poupança, indicando na coluna Saque o valor da tranferência.</li>
</ol>
- O que mais?

É só isso! Só lembrando que a dica é de personalizar cada conta. Apesar desta série seguir os nomes que vêm por padrão no GNUCash, por questão de facilidade.

Na próxima página: Aposentadoria Privada, Juros e Freelance.

<!--nextpage-->

<h3>Aposentadoria Privada</h3>
- É, eu penso no meu futuro.

Um plano de aposentadoria privada pode ser um grande diferencial para uma velhice mais tranquila. O meu plano é associado com a minha conta poupança. Assim, todo mês é debitado um determinado valor, referente ao plano que escolhi, da minha conta poupança. Esse valor vai para a minha conta da aposentadoria privada e fica por lá, como um investimento independente da poupança.

Os passos para realização desta transação são semelhantes aos já explicados neste texto:
<ol>
    <li>expanda a sua conta poupança.</li>
    <li>crie uma transação para Ativos:Investimentos:Aposentadoria:Fundo de Investimentos no valor do depósito - indicado na coluna Saque.</li>
</ol>
O lançamento foi feito. Pode expandir a árvore de contas e conferir.
<h3>Juros</h3>
- Como eu lanço os juros que receber?

Poupança e Aposentadoria Privada são dois tipos de investimentos que rendem juros. Para fazer o lançamento destes valores, use a conta Receitas:Juros Recebidos. Siga os passos:
<ol>
    <li>expanda Receitas:Juros Recebidos:Juros Conta Poupança; e</li>
    <li>crie uma transação para Ativos:Ativos Atuais:Conta Poupança no valor dos juros recebidos, indicando o valor na coluna Receita.</li>
</ol>
Pronto! Para fazer o lançamento dos juros de aposentadoria privada, os passos são semelhantes, mas a origem dos juros pode ser a conta Receitas:Juros Recebidos. Outra alternativa é criar uma subconta dentro de Receitas:Juros Recebidos com o nome de Juros Aposentadoria Privada e fazer o lançamento de lá.
<h3>Freelance</h3>
- Esse mês eu fiz uns trabalhinhos por fora. Como posso lançá-los no GNUCash?

Há duas formas. Se você não faz freelances sempre, pode fazer o lançamento de Receitas:Outras Receitas. Se você fizer freelances com mais frequência, sugiro a criação da conta Receitas:Outras Receitas:Freelances e lançar de lá, por questão de organização e melhor discriminação da origem do dinheiro.

Seja qual for a sua opção basta realizar uma transferência para a conta de destino do dinheiro, indicando o valor recebido na coluna Receita.
