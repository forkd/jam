rc="http://joselop.es/app/wor/wp-content/uploads/2010/12/conversation_with_smaug-231x300.jpg" alt="Conversation with Smaug - Concept Art" title="Conversation with Smaug - Concept Art" width="231" height="300" class="alignright size-medium wp-image-167" />Emprestar dinheiro não é mau negócio. Isto é, se você não tiver uso para o mesmo e se tiver certeza de que será pago. Se for receber juros, melhor ainda - acho que há uma lei sobre agiotagem que regula isso. Seja como for, é preciso manter o controle deste dinheiro para poder cobrar e mesmo para poder projetar futuros investimentos.

Para gerenciar os empréstimos a receber, eu recomendo que se crie a conta Ativos:Contas a Receber. Com essa conta criada, as operações ficam simples. Imagine o caso dos três empréstimos citados anteriormente. Para lançá-los, crie transações entre Ativos:Contas a Receber e a conta de onde saiu o dinheiro. Sugiro ainda colocar na descrição algo como "Empréstimo Miguel", por exemplo. Os lançamentos dos valores devem ser indicados na coluna Aumentar.

Para receber os pagamentos, faça uma transação entre Ativos:Contas a Receber e a conta para onde o dinheiro foi - gosto de descrever esse tipo de transações como "Pagamento Ana", por exemplo. Lembre-se de indicar o valor na coluna Decrescer. Repita isso para todos os pagamentos que receber.

Se for receber juros, os mesmos podem ser lançados como transferências entre Receitas:Juros Recebidos:Outros Juros e a conta para onde o dinheiro foi - coluna Receita. Eu costumo colocar algo como "Juros Pagamento André" na descrição deste tipo de operações.

Caso você empreste dinheiro para uma pessoa com frequência, ou caso o valor seja muito alto, pode ser necessário descrever melhor esse empréstimo especificamente. Para isso, crie, dentro de Ativos:Contas a Receber, uma subconta com o nome da pessoa para quem emprestou. Faça todas as transações a partir desta subconta, pois facilitará muito o gerenciamento deste empréstimo.

<h3>Pegando Dinheiro Emprestado</h3>

- Preciso de dinheiro para investir no meu negócio.

Pegar dinheiro emprestado nem sempre é uma boa ideia. Contudo, às vezes é necessário para poder melhorar seu empreendimento, comprar um automóvel, casa, enfim... Lembre-se de ficar sempre atento às taxas de juros e condições de pagamento, OK? Após analisar todas as possibilidades, se for pegar mesmo o empréstimo, lance-o no GNUCash. Isso facilitará o gerenciamento dos pagamentos.

Imagine que você tenha pegado R$ 200.000,00 para a compra de uma casa. Lance isso no GNUCash como uma transação entre Passivo:Contas a Pagar e Ativos:Ativos Fixos:Casa. Como o valor é alto, pode ser interessante criar a subconta Casa dentro de Passivo:Contas a Pagar. Note também que a casa em questão é tratada como um ativo, por ser um bem de consumo durável. Isso vem por padrão no GNUCash e eu preferi manter assim - apesar do Robert Kiyosaki poder discordar de mim.

Para pagar este empréstimo, realize transações nos valores das parcelas entre a conta que vai sair o dinheiro (e.g., Ativos:Ativos Atuais:Conta Corrente) e Passivo:Contas a Pagar. Indique nessas transações o número da parcela em descrições como Pagamento Empréstimo Casa 1. O valor desta transação deverá ser indicado na coluna Saque.
