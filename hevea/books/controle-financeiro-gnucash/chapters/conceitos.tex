ass="alignright size-medium wp-image-162" title="Money Heart - R7" src="http://joselop.es/app/wor/wp-content/uploads/2010/12/money_heart-255x300.jpg" alt="Money Heart - R7" width="255" height="300" />Muitas pessoas não sentem necessidade de controlar o destino que o seu dinheiro toma. Se sobra dinheiro no final do mês, pode-se gastar naquele bar bacana ou então depositar na poupança. Se não sobra, basta aguardar o próximo mês que tem mais chegando. Se falta, tira da poupança...

Apesar de ser fácil, o modelo apresentado não é sustentável. Você não terá controle de onde estão os seus gastos. Caso decida que é necessário economizar, nem saberá onde deverá ou poderá cortar. Além disso, no pior caso, que é faltar dinheiro no final do mês e recorrer à poupança, você corre o risco de começar um ciclo onde gastará mais do que ganha. Ora, acho que todos sabemos que a poupança não rende muito e se fosse assim, uma hora o dinheiro de lá acabaria. O que você faria então? Pediria emprestado? Já pensou nos juros absurdos que são cobrados pelas financeiras e pelos bancos?

Pior disso tudo é que talvez você esteja gastando seu rico dinheirinho em coisas supérfluas. Em outras palavras: bastaria racionalizar o uso do dinheiro para terminar o mês no verde. Sem recorrer à poupança, sem empréstimos e talvez ainda guardando um pouco! Por isso é necessário controlar o uso do seu dinheiro.
<h3>Softwares de Controle Financeiro</h3>
Se você pesquisar no Oráculo, verá que há vários softwares para auxiliá-lo no seu controle financeiro. Alguns são pagos, outros são livres. Alguns têm milhares de funções, sendo utilizados até por empresas e escritórios de contabilidade. Outros são tão simples que vêm como planilhas. Alguns são tão complexos que você precisaria de uma faculdade de contabilidade, mais 1 mês de treinamento com os desenvolvedores. Outros são tão crus, que o seu sobrinho de 11 anos já fez melhores.

Então você, um pobre ser humano que não sabe nadar, se vê sozinho num mar de softwares de gestão financeira e não sabe qual escolher. Eu escolhi o <a title="GNUCash" href="http://www.gnucash.org/" target="_blank">GNUCash</a>. Por quê?
<ul>
    <li>Software gratuito e livre.</li>
    <li>Multiplataforma.</li>
    <li>Simples o bastante para leigos na área contábil.</li>
    <li>Complexo o bastante para manter um controle detalhado do fluxo monetário.</li>
</ul>

- Já sei da necessidade em controlar onde gasto meu dinheiro e que o GNUCash é uma boa escolha para me ajudar nesse processo!

Ótimo! Já que te convenci dessas duas coisas, vamos aos conceitos que você precisará para fazer o seu controle financeiro.
<h3>Conceitos</h3>
Quero deixar claro que estas são as formas com que eu entendo estes conceitos. Pode ser que um contador, economista, ou mesmo outro leigo como eu, não concorde. Pode ser que a literatura especializada defina de outra forma. Esta é a forma que funciona para mim e por isso a estou descrevendo neste texto. Afinal, a ideia não é formar contadores com esta série, mas ajudar leigos a controlar melhor suas finanças.

[tabs tab1="Bens de Consumo" tab2="Receita" tab3="Despesas" tab4="Ativos" tab5="Passivos" tab6="Transações"]
[slide1]
São bens destinados a satisfazer as suas necessidades e/ou da sua família. Subdividem-se em 3 tipos:
<ol>
    <li>Não Duráveis - são feitos para serem consumidos imediatamente, e.g., lanches, sucos, produtos de limpeza.</li>
    <li>Semi Duráveis - são bens que se desgastam e você não permanece com eles por muito tempo, e.g., roupas, calçados, acessórios.</li>
    <li>Duráveis - são aqueles que foram feitos para serem usados por longos períodos, e.g., imóveis, automóveis, eletrônicos.</li>
</ol>
[/slide1]

[slide2]
Esqueça bolos e o livrinho da sua avó. Estamos falando de onde vem o seu dinheiro.
<ul>
    <li>Salário?</li>
    <li>Freelance?</li>
    <li>Mesada?</li>
    <li>Presente?</li>
</ul>
Tanto faz. Esses são exemplos de origem do seu dinheiro. São tipos de receitas.
[/slide2]

[slide3]
São custos operacionais para manutenção dos bens e da qualidade de vida. Normalmente vêm sob a forma de contas, e.g., contas de água, energia e Internet, manutenção de casa e automóvel, gastos com saúde.
[/slide3]

[slide4]
Calma! Este texto não tem teor sexual!

Quando falamos em ativos no escopo de finanças, estamos nos referindo ao conjunto de bens, valores e créditos que compõem o seu patrimônio. É tudo o que pode ser convertido em dinheiro, e.g., contas bancárias, poupança, ações, fundos de investimento, empréstimos a receber.

[/slide4]

[slide5]
São o total de dívidas e obrigações que você possui, e.g., fatura do cartão de crédito, empréstimos a pagar, compras parceladas.
[/slide5]

[slide6]
No contexto do GNUCash, tudo são transações. Isso se dá porque ele funciona de acordo com <a title="Wikipédia - Método de Partida Dobrada" href="http://pt.wikipedia.org/wiki/M%C3%A9todo_das_partidas_dobradas" target="_blank">método de partida dobrada de Luca Pacioli</a>. Este método nos diz, resumidamente, que toda operação financeira é uma transação entre contas. e.g., ao retirar dinheiro da sua Conta Corrente, você estaria realizando uma transação do tipo Ativos:Conta Corrente para Ativos:Dinheiro na Carteira; ao comprar um tênis,  a transação seria Ativos:Conta Corrente para Despesas:Vestuário.

Essas transações são úteis para saber os caminhos que o nosso dinheiro toma. Sabendo desses caminhos, você estará apto a decidir o que é essencial e o que é supérfluo.

[/slide6]
[/tabs]
