ass="alignright size-medium wp-image-164" title="Money Elephant - Wendy's Origami" src="http://joselop.es/app/wor/wp-content/uploads/2010/12/money_elephant-300x300.jpg" alt="Money Elephant - Wendy's Origami" width="300" height="300" />Suponha que você tenha várias contas correntes. Seria interessante alterar o nome de cada uma para saber exatamente de qual conta se trata.

Eu, por exemplo, gosto de alterar os nomes das minhas contas correntes e das minhas contas poupanças para "Banco - Número" - e.g., Bradesco - 9999-999999-9. Dessa forma eu consigo descobrir facilmente a qual conta eu estou me referindo em uma transação.

Para editar um conta, clique com o botão direito em cima dela e selecione a opção Editar conta no menu de contexto - pode-se também selecionar a conta e clicar no botão Editar a conta selecionada, na Barra de Ferramentas. Na janela que se abrir, altere os valores de acordo com a necessidade e clique no botão OK.
<h3>Adição de Contas</h3>
- Eu adicionei todas as contas possíveis na primeira execução do GNUCash!

Tudo bem, mas acredite em mim: haverá casos onde será necessário adicionar mais contas. Imagine que você possua 2 contas poupanças.

Mesmo adicionando todas as contas na primeira utilização do GNUCash (<a title="GNUCash: Básico" href="http://joselop.es/gnucash-basico/" target="_self">GNUCash: Básico</a>), você possuirá apenas uma conta poupança para lançamento de dados. Apesar de poder discriminar na transação que determinado valor é para a poupança tal, é sempre mais interessante descrever melhor o caminho do dinheiro. Dessa forma torna-se necessária a adição de novas contas.

Clique com o botão direito do mouse sobre a conta à qual a nova conta pertencerá e selecione a opção Nova Conta. Na janela que se abre, preencha os campos Nome da conta e Descrição, clicando em OK ao final.
<h3>Remoção de Contas</h3>
Da mesma forma com que a adição de novas contas ajuda na definição do caminho do dinheiro, há contas que não são usadas por muitas pessoas - e.g., se você não possui e nem pretende ter um plano de aposentadoria privada, não precisa das contas destinadas à essa aplicação. Por isso é interessante remover contas sem uso.

Clique com o botão direito do mouse sobre a conta a ser removida e selecione a opção Excluir Conta, confirmando em seguida - você também pode utilizar a Barra de Ferramentas para isso.)
