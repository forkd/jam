% Hevea - Criptografia numa Caixa de Fósforos
% An article --pt-br-- about cryptography.
%
% Author: José Lopes de Oliveira Jr. <jilo.cc>
%
% LICENSE
% This program is free software: you can redistribute it and/or modify
% it under the terms of the GNU General Public License as published by
% the Free Software Foundation, either version 3 of the License, or
% (at your option) any later version.
%
% This program is distributed in the hope that it will be useful,
% but WITHOUT ANY WARRANTY; without even the implied warranty of
% MERCHANTABILITY or FITNESS FOR A PARTICULAR PURPOSE.  See the
% GNU General Public License for more details.
%
% You should have received a copy of the GNU General Public License
% along with this program.  If not, see <http://www.gnu.org/licenses/>.
%%


%%
% PREAMBLE
%
\documentclass[12px,a4paper,twoside]{article}

%%
% DRAFT MARK
% Comment these lines on final version.
%\usepackage{draftwatermark}
%\SetWatermarkText{DRAFT}
%\SetWatermarkScale{4}

%%
% GENERAL PACKAGES
%
\usepackage{graphicx}   %for including pictures
\usepackage{fixltx2e}   %for \textsubscript{}
\usepackage{amssymb}    %special math symbols

%%
% ENCODING & LOCALE
%
\usepackage[utf8]{inputenc}    %utf-8 encoding
\usepackage[T1]{fontenc}       %extended font encoding
\usepackage[brazilian]{babel}  %pt-BR locale

%%
% PDF PROPERTIES
%
\usepackage[pdftitle={Criptografia numa Caixa de Fósforos},
            pdfauthor={José Lopes de Oliveira Jr.},
            pdfsubject={Notas de estudo em criptografia.},
            pdfkeywords={criptografia, segurança da informação,
                         certificação digital, GnuPG}]{hyperref}

%%
% LISTINGS
% To better present source-codes.
\usepackage{listings}
\lstset {
    inputencoding=utf8,
    extendedchars=true,
    language=Python,
    tabsize=4,
    basicstyle=\footnotesize\ttfamily,
    stringstyle=\em,
    numbers=left,
    numberstyle=\tiny,
    stepnumber=1,
    numbersep=10pt,
    frame=single,
    showstringspaces=false,
    literate={á}{{\'a}}1    {Á}{{\'A}}1  % pt_BR accented chars
             {à}{{\`a}}1    {À}{{\`A}}1
             {ã}{{\~a}}1    {Ã}{{\~A}}1
             {â}{{\^a}}1    {Â}{{\^A}}1
             {é}{{\'e}}1    {É}{{\'E}}1
             {í}{{\'i}}1    {Í}{{\'I}}1
             {ó}{{\'o}}1    {Ó}{{\'O}}1
             {õ}{{\~o}}1    {Õ}{{\~O}}1
             {ô}{{\^o}}1    {Ô}{{\^O}}1
             {ú}{{\'u}}1    {Ú}{{\'U}}1
             {ü}{{\"u}}1    {Ü}{{\"U}}1
             {ç}{{\c{c}}}1  {Ç}{{\c{C}}}1
}

%%
% DOCUMENT PROPERTIES
%
\title{Criptografia numa Caixa de Fósforos}
\author{José Lopes de Oliveira Jr.\\
{\small \href{mailto:jlojunior@gmail.com}{jlojunior@gmail.com}}}


%%%%%%%%%%%%%%%%%%%%
% BODY
%
\begin{document}
\maketitle
\begin{abstract}
\noindent Tão importante quanto conhecer os conceitos da criptografia, é saber 
como colocá-los em prática e suas fragilidades.  Este texto objetiva apresentar 
essas questões com exemplos usuais, de forma a dar uma visão mais cotidiana 
sobre o assunto.  Espera-se que o leitor torne-se apto a identificar quais 
sistemas criptográficos pode usar e quais práticas deve adotar para melhorar ou 
manter o nível de segurança esperado para suas informações.
\newline\newline
\noindent \textbf{Palavras-chave:} criptografia, segurança da informação, 
certificação digital, GnuPG.
\end{abstract}


\section{Introdução}
\label{sec:intro}

A criptografia é definida por \cite{uchoa} como a arte e ciência de manter 
mensagens seguras.  De acordo com o mesmo autor, sistemas criptográficos são 
necessários para evitar uma série de problemas de espionagem nas comunicações 
eletrônicas.

Este texto explorará a teoria e a prática acerca da criptografia, englobando 
testes de segurança sobre arquivos criptografados.  Na seção \ref{sec:teoria}
serão apresentados os tipos de algoritmos criptográficos existentes, bem como
sistemas complementares a eles.  Na seção \ref{sec:pratica} será mostrado como
utilizar os conceitos apresentados, usando o GnuPG.  Já na seção
\ref{sec:ataques} serão mostradas algumas formas de ataque sobre a criptografia
e na última seção, serão apresentadas as conclusões obtidas neste trabalho,
seguidas de ideias para trabalhos futuros e as referências utilizadas.


\section{Teoria}
\label{sec:teoria}

Nesta seção serão descritas as premissas acerca da criptografia, abrangendo os
principais algoritmos e sistemas complementares ou derivados deles.

\subsection{Criptografia}
\label{sec:teoria:criptografia}
De acordo com \cite{tanenbaum:1}, os métodos criptográficos seguem o padrão das
equações:

\begin{equation}
\label{eq:crypt}
    C = E(M, K_e)
\end{equation}
\begin{equation}
\label{eq:decrypt}
    M = D(C, K_d)
\end{equation}

Em (\ref{eq:crypt}) tem-se que a mensagem criptografada ($C$) é obtida pelo
processo de cifragem ($E$) da mensagem ($M$), usando a chave de cifragem
($K_e$).  Já em (\ref{eq:decrypt}) tem-se que a mensagem será novamente obtida
aplicando-se o processo de decifragem ($D$) à mensagem cifrada, usando-se a
chave de decifragem ($K_d$).

\cite{uchoa} define que existem duas classes de algoritmos: os de chave privada
ou simétricos e os de chave pública ou assimétricos e, como encontrado em
\cite{tanenbaum}, o Princípio de Kerckhoff, afirma que uma premissa importante
é que os algoritmos devem ser públicos e as chaves secretas.

\subsection{Algoritmos de Chave Privada (Simétricos)}
\label{sec:teoria:simetricos}
Estes algoritmos usam uma mesma chave para criptografar e decriptografar e por
isso a chave deve ser mantida em segredo \cite{uchoa}.  Algoritmos simétricos
costumam usar chaves de poucos bits para fazer a criptografia, mas para
melhorar a segurança, pode-se usar chaves de 1024 bits \cite{tanenbaum:1}.
Ainda segundo a mesma fonte, são algoritmos eficientes pois a computação
necessária para criptografar/decriptografar uma mensagem é controlável e sua
maior desvantagem é que o emissor e o receptor devem possuir a chave secreta
compartilhada e a mesma precisa ser transmitida de alguma forma, com segurança.

\subsubsection{Data Encryption Standard (DES)}
\label{sec:teoria:simetricos:des}
Criado pela IBM e adotado pelo Governo dos Estados Unidos da América (EUA),
este algoritmo não é mais seguro em sua forma original.  Gera um texto cifrado
de 64 bits e utiliza chaves de 56 bits.  A chave é aplicada ao texto em 16
iterações \cite{tanenbaum}.

\subsubsection{3DES (Triple DES)}
\label{sec:teoria:simetricos:3des}
Conforme \cite{tanenbaum}, é uma variante do DES que usa duas chaves, $K_1$ e
$K_2$, de 56 bits. Realiza um processo conhecido por
\textit{Encryption-Decryption-Encryption} (EDE) para criptografar e
\textit{Decryption-Encryption-Decryption} (DED) para decriptografar, aplicando
$K_1$, $K_2$, $K_1$, respectivamente em cada estágio.  O esquema EDE/DED foi
usado para garantir compatibilidade com o DES original, definindo $K_1 = K_2$.

\subsubsection{Advanced Encryption Standard (AES)}
\label{sec:teoria:simetricos:aes}
O AES surgiu da necessidade de substituir o DES.  O Governo dos EUA promoveu um
concurso público para selecionar o algoritmo ---a fim de dar transparência ao
processo--- e o escolhido foi o Rijndael \cite{tanenbaum}.

\cite{tanenbaum} afirma que o AES permite tamanhos de chaves e blocos de 128 a
256 bits, com intervalos de 32 bits, sendo os comprimentos da chave e do bloco
independentes.  Contudo, especifica que o bloco deve ter 128 bits e as chaves,
128, 192 ou 256 bits, sendo que costuma-se usar 128 ou 256 bits apenas.

Segundo \cite{tanenbaum}, tanto o DES quanto o AES utilizam cifras de
substituição monoalfabéticas. Assim, ao processar o mesmo bloco de texto com a
mesma chave, o mesmo resultado será obtido.  Para contornar este problema,
conforme a mesma fonte, os modos \textit{electronic code block} (ECB), de
encadeamento de blocos de cifras, de cifra de fluxo e de contador foram
criados. Cada um com vantagens e desvantagens que os tornam aptos para
determinadas situações.

Outros algorimos de chave simética são Blowfish, IDEA, RC4, RC5, Serpent e
Twofish.

\subsection{Algoritmos de Chave Pública (Assimétricos)}
\label{sec:teoria:assimetricos}
\cite{tanenbaum} afirma que o maior problema da criptografia de chave simétrica
é manter segredo sobre a chave secreta, pois ela precisa ser distribuída, mas
deve ser confidencial.  Assim, Whitfield Diffie e Martin Hellman propuseram, em
1976, um método radicalmente novo, onde:

\begin{equation}
M = D(E(M))
\end{equation}

Algoritmos com esse formato foram definidos como assimétricos ou de chave
pública, pois utilizam uma chave para criptografar e outra para decriptografar
\cite{uchoa}.  Normalmente a chave de criptografia é chamada de chave pública e
a de decriptografia, de chave privada, mas há casos, como na assinatura digital
---seção \ref{sec:teoria:assinaturas}---, em que é necessário inverter essa
ordem ---chave privada para criptografar e chave pública para decriptografar.
Como \cite{tanenbaum} explica, o conhecimento de uma chave a partir da outra é
extremamente difícil.

Esses algoritmos são mais lentos que os simétricos pois utilizam operações
fáceis, como multiplicação, para criptografar ---normalmente alimentada por uma
senha como semente para o algoritmo--- e operações difíceis para decriptografar
\cite{tanenbaum}.

Algoritmos de chave pública são amplamente utilizados para distribuir a chave
de sessão, que passa a ser usada por algoritmos simétricos, como o DES ou o
AES, mais rápidos \cite{tanenbaum}.  Tal método pode ser observado no
funcionamento do \textit{Transport Layer Security} (TLS), por exemplo.

Conforme \cite{tanenbaum}, o RSA é o algoritmo assimétrico mais famoso e é
muito forte.  Ele utiliza chaves de pelo menos 1024 bits, o que o torna bem
mais lento que os algoritmos de chave simétrica e sua segurança é baseada no
fato de ser extremamente difícil fatorar números primos extensos ---geralmente
1024 bits.

Segundo \cite{uchoa}, a dinâmica dessa classe de algoritmos é que a pessoa
divulga sua chave pública e sempre que alguém precisa, criptografa uma mensagem
com essa chave e envia para a pessoa, que utiliza sua chave privada para
decriptografar.

Outros exemplos de algoritmos assimétricos são ElGamal e Digital Signature
Standard (DSS).

\subsection{Sumários de Mensagens ---\textit{Message Digests}}
\label{sec:teoria:sumarios}
É, como definido por \cite{tanenbaum}, uma função de hash unidirecional
---comumente representada por $MD$--- que gera uma saída de tamanho fixo a
partir de uma mensagem de tamanho variável.  Essa função tem quatro
características:

\begin{enumerate}
    \item tendo $M$ é fácil calcular $MD(M)$;
    \item tendo $MD(M)$ é impossível, efetivamente, encontrar $M$;
    \item tendo $M$ não se pode achar $M | MD(M') = MD(M)$; e
    \item uma mudança, mesmo de 1 bit, produz $MD(M)$ diferente.
\end{enumerate}

Ainda segundo \cite{tanenbaum}, para a terceira característica, a função deve
ter 128 bits ou mais.  Para atender à quarta característica, a função deve
embaralhar os bits, como os algoritmos de chave simétrica.

Dois algoritmos notórios são Message Digest 5 (MD5) e Secure Hash Algorithm
(SHA) \cite{tanenbaum:1}.

\subsection{Assinaturas Digitais}
\label{sec:teoria:assinaturas}
\cite{tanenbaum} explica que assinatura digital é o método usado para verificar
a integridade e evitar o não-repúdio de documentos públicos, impossibilitando
fraudes no documento original.

Ainda de acordo com \cite{tanenbaum}, o algoritmo para geração de uma
assinatura digital começa com cálculo do hash do documento com um algoritmo
específico, como MD5 ou SHA-1.  Depois criptografa-se esse hash com a chave
privada do emissor e envia-se o resultado junto com o documento.

O receptor, ao receber os dados, calcula o hash do documento e aplica a chave
pública do emissor sobre o resultado enviado, obtendo o hash gerado pelo
emissor.  Se os dois hashes forem iguais, tudo está correto.

Este é um método viável pois a criptografia de chave pública ---lenta--- é
aplicada apenas a uma pequena quantidade de bits ---o hash \cite{tanenbaum:1}.

\subsubsection{Assinaturas de Chave Simétrica}
\label{sec:teoria:assinaturas:simetricas}
Segundo \cite{tanenbaum}, necessitam da presença de uma autoridade central e
confiável.  Assim, as pessoas geram suas chaves e se cadastram junto a essa
autoridade.  Sempre que for necessário enviar uma mensagem assinada para
alguém, a pessoa criptografa sua mensagem com a chave gerada e a envia para a
autoridade, junto com dados com o timbre e hora.  A autoridade decriptografa a
mensagem para ter certeza que é do remetente e então a reenvia como texto
simples ao destinatário, junto com a sua assinatura, para dar veracidade ao
documento.

A desvantagem desse método é que todos devem confiar plenamente na autoridade
certificadora, que também poderá ler todas as mensagens que for encaminhar
\cite{tanenbaum}.

\subsubsection{Assinaturas de Chave Pública}
\label{sec:teoria:assinaturas:assimetricas}
De acordo com \cite{tanenbaum:1}, baseia-se na propriedade de comutação do
algoritmo, ou seja, obter a mensagem a partir da criptografia da decriptografia
e vice-versa:

\begin{equation}
M = E(D(M))
\end{equation}
\begin{equation}
M = D(E(M))
\end{equation}

Assim, se Alice\footnote{\cite{oliveira:2} explica que a literatura adotou os
nomes Alice e Bob para tornar mais didáticos os exemplos que usam algo como
``criptografado por A e decriptografado por B.''} quiser enviar uma mensagem
assinada para Bob, ela deverá ``criptografar'' a mensagem com sua chave privada
e criptografar o resultado com a chave pública de Bob ---$E_b(D_a(M))$.  Quando
Bob receber a mensagem, usará sua chave privada para decriptografar, aplicando
a chave pública de Alice para ``decriptografar'' o resultado ---$D_b(E_a(M))$.
Isso irá gerar a mensagem oriunda de Alice.

\subsection{Gerenciamento de Chaves Públicas}
\label{sec:teoria:pki}
Conforme \cite{tanenbaum}, é uma forma de se disponibilizar chaves públicas em
um repositório.  Neste contexto surge o \textit{Certification Authority} (CA)
como agente que certifica chaves públicas ---garante que o seu dono é quem diz
ser e que a chave não foi alterada por uma terceira parte.

O X.509 é um padrão para certificados que define, dentre outras coisas, vários
campos que um certificado enviado a uma CA deve conter \cite{tanenbaum}.

Como não é interessante ter uma única CA no mundo e nem ter várias CAs, foi
criada a \textit{Public Key Infrastructure} (PKI): um modo de estruturar
componentes e definir padrões para os vários documentos e protocolos.  Neste
caso, a CA de nível superior ---raíz--- certifica CAs de segundo nível
---\textit{Regional Authorities} (RAs)---, que certificam CAs reais que emitem
por sua vez, os certificados X.509 \cite{tanenbaum}.

\cite{tanenbaum} ainda pondera que a autoridade que concede um certificado pode
revogá-lo.  Há várias possibilidades de implementação disso, que passam por uma
\textit{Certification Revogation List} (CRL).  Esta lista pode ser atualizada
periodicamente com a situação de cada certificado, somente com os certificados
válidos etc., dependendo da forma com que for implementada.


\section{Prática}
\label{sec:pratica}
O GNU Privacy Guard (GnuPG) é, segundo \cite{gnupg:hp}, uma implementação
completa e livre do padrão OpenPGP, como definido na RFC 4880.  O comando
\texttt{gpg}, que será utilizado neste artigo, de acordo com \cite{gnupg:man},
é a parte OpenPGP do GnuPG: uma ferramenta para prover criptografia digital e
serviços de assinatura usando o padrão OpenPGP e que provê funcionalidades
completas para o gerenciamento de chaves criptográficas.

Os conceitos da seção \ref{sec:teoria} serão aplicados com o GnuPG nesta seção.
Não é objetivo deste texto ser, portanto, um manual definitivo para o programa.
Caso o leitor queira saber mais opções e exemplos de utilização do GnuPG,
recomenda-se a leitura de \cite{oliveira:1} e do seu próprio manual, em
\cite{gnupg:man}.  A versão do programa utilizada nos testes foi a 1.4.11,
juntamente com o arquivo da RFC
4880\footnote{\url{http://www.ietf.org/rfc/rfc4880.txt}}.

\subsection{Criptografia de Chaves Simétricas}
\label{sec:pratica:simetrica}
O primeiro teste realizado foi criptografar o arquivo, usando o AES com 256
bits, para gerar um arquivo binário criptografado ---\texttt{rfc4880.txt.gpg}:

\begin{verbatim}
$ gpg --symmetric --cipher-algo aes256 rfc4880.txt
\end{verbatim}

O arquivo binário não é, teoricamente, legível por humanos.  Para gerar um
arquivo compreensível, pode-se usar a opção \texttt{-{}-armor}, como em:

\begin{verbatim}
$ gpg --armor --symmetric --cipher-algo aes256 rfc4880.txt
\end{verbatim}

O comando de decriptografia é o mesmo para ambas as saídas geradas
---\texttt{.gpg} ou \texttt{.asc}:

\begin{verbatim}
$ gpg rfc4880.txt.asc
\end{verbatim}

O interessante de se gerar um arquivo texto é fazer testes com ele.  Pode-se,
por exemplo, inserir ou remover trechos no mesmo e tentar decriptografar.
Pode-se ainda trocar algumas partes do texto, salvar e observar a falha na
decriptografia.  Então pode-se voltar para a versão original e constatar que a
decriptografia ocorre com sucesso ---o GnuPG não leva em consideração o
\textit{timestamp} do arquivo.

\subsection{Criptografia de Chaves Assimétricas}
\label{sec:pratica:assimetrica}
A primeira etapa de testes foi de geração das chaves para Alice e Bob.  Para
Alice:

\begin{verbatim}
$ gpg --gen-key
\end{verbatim}

\begin{enumerate}
    \item \texttt{enter} para gerar chaves RSA ---padrão na versão utilizada.
    \item \texttt{enter} para usar chaves de 2048 bits ---padrão.
    \item \texttt{enter} para as chaves não expirarem ---padrão. \texttt{y}
        para confirmar.
    \item \begin{itemize}
            \item \textit{Real name:} \texttt{Alice}
            \item \textit{Email address:} \texttt{alice@example.com}
            \item \textit{Comment:} \texttt{enter}.
            \item \textit{Passphrase:} \texttt{alice}
          \end{itemize}
    \item Entropia. Deve-se digitar algo no teclado e/ou mover o mouse para
        gerar entropia para as chaves.
\end{enumerate}

Para Bob os passos foram iguais, apenas os dados do passo 4 diferiram:

\begin{itemize}
    \item \textit{Real name:} \texttt{Robert}
    \item \textit{Email address:} \texttt{bob@example.com}
    \item \textit{Comment:} \texttt{enter}.
    \item \textit{Passphrase:} \texttt{bob}
\end{itemize}

Para exportar as chaves públicas de Alice e Bob, pode-se usar o parâmetro
\texttt{-{}-export} do \texttt{gpg}:

\begin{verbatim}
$ gpg --armor --export alice
$ gpg --armor --export bob
\end{verbatim}

Sem a opção \texttt{-{}-armor}, será exportado um arquivo binário do GnuPG.
Para direcionar a saída para um arquivo, utiliza-se a opção \texttt{-{}-output}
seguida do nome do arquivo:

\begin{verbatim}
$ gpg --output alice-pubkey.gpg --export alice
\end{verbatim}

Para exportar a chave privada, troca-se \texttt{-{}-export} por
\texttt{-{}-export-secret-key}:

\begin{verbatim}
$ gpg --output alice-seckey.asc --armor --export-secret-key alice
\end{verbatim}

É importante usar esta opção com cautela e lembrar-se da grande
responsabilidade que se deve ter sobre o sigilo da chave privada.

Para importar uma chave pública, basta usar a opção \texttt{-{}-import}:

\begin{verbatim}
$ gpg --import bob-pubkey.asc
\end{verbatim}

Pode-se imaginar um cenário onde Alice deseja enviar o arquivo
\texttt{rfc4880.txt}  de forma segura para Bob, usando criptografia
assimétrica.  Para isso ela deverá criptografá-lo com a chave pública de Bob:

\begin{verbatim}
$ gpg --encrypt --recipient bob rfc4880.txt
\end{verbatim}

Isso vai gerar o arquivo binário criptografado \texttt{rfc4880.txt.gpg}.  Para
gerar um arquivo texto, pode-se incluir a opção \texttt{-{}-armor} como
argumento para o GnuPG.  Neste caso, será gerado um arquivo com a extensão
\texttt{.asc}.  Este arquivo poderia ser enviado para Bob por email, por
exemplo.

Ao receber o arquivo, Bob deve aplicar sua chave privada no mesmo para
decriptografá-lo:

\begin{verbatim}
$ gpg --output rfc4880.txt --decrypt rfc4880.txt.gpg
\end{verbatim}

Ele digitaria a senha que usou para gerar suas chaves ---\texttt{bob}--- e o
arquivo \texttt{rfc4880.txt} seria criado com o texto original.  Interessante
notar que, caso o parâmetro \texttt{-{}-output} e seu argumento sejam omitidos,
o GnuPG direcionará o texto decriptografado para a saída padrão --- a tela, na
maioria dos casos.  Outra saída possível seria omitir esses dados e
redirecionar a saída para um arquivo.

Esse esquema é muito útil quando autenticação e sigilo são necessários, pois
Alice teria certeza de que o texto seria lido apenas por Bob.  Ele, por sua
vez, poderia saber se foi Alice mesmo quem enviou, apenas criptografando uma
mensagem com a chave pública dela e enviando.

\subsection{Assinaturas Digitais}
\label{sec:pratica:assinaturas}
Bob precisaria saber que o \textit{script} a ser publicado no site foi
realmente enviado por Alice, mas o seu conteúdo não é segredo para ninguém.
Neste caso entra o conceito de assinatura digital.  Como Alice não requer
sigilo sobre o arquivo para Bob, ela assinaria o arquivo com:

\begin{verbatim}
$ gpg --local-user alice --sign rfc4880.txt
\end{verbatim}

Seria então pedida a senha que Alice usou na criação da sua chave, para assinar
o documento.  O parâmetro \texttt{-{}-local-user} pode ser omitido caso haja
apenas um usuário cadastrado no GnuPG ---arquivo
\texttt{$\sim$/.gnupg/pubring.gpg}.  Caso ele seja usado, como neste exemplo,
buscará pela primeira ocorrência que casar com o argumento passado
---\texttt{alice}, no caso.  Se houver ambiguidade neste nome, pode-se usar o
identificador da chave, e.g., \texttt{e81676ca} ---lembrando que o GnuPG
\textbf{não} é \textit{case-sensitive}.  O comando mostrado geraria o arquivo
binário \texttt{rfc4880.txt.gpg}. Para gerar um arquivo separado, substitui-se
a opção \texttt{-{}-sign} por \texttt{-{}-clearsign}, que resultaria no arquivo
\texttt{rfc4880.txt.asc}.

Se um arquivo texto for gerado, basta listá-lo para conferir que o texto está
legível.  Contudo, ao final do mesmo, terão sido adicionadas algumas linhas com
a versão do GnuPG usada para assinar e a assinatura propriamente dita.  Para
gerar a assinatura em um arquivo separado, pode-se substituir a opção
\texttt{-{}-sign} ---ou \texttt{-{}-clearsign}--- por \texttt{-{}-detach-sign}.
Usado dessa forma, o GnuPG criará um arquivo binário com a extensão
\texttt{sig}.  Caso seja interessante gerar um arquivo texto, pode-se usar a
opção \texttt{-{}-armor} na chamada do GnuPG, como no próximo trecho de código:

\begin{verbatim}
$ gpg --armor --local-user alice --detach-sign rfc4880.txt
\end{verbatim}

Isso vai gerar o arquivo \texttt{rfc4880.txt.asc}, que se for listado mostrará
a assinatura do arquivo.

Ao receber o arquivo de Alice, Bob poderá verificá-lo com o comando:

\begin{verbatim}
$ gpg --verify rfc4880.txt.gpg
\end{verbatim}

Obviamente isto já leva em consideração que Bob tenha importado a chave pública
de Alice para o seu sistema.  Caso Alice tenha gerado a assinatura em um
arquivo separado, basta Bob passar o arquivo com a assinatura e o arquivo com o
texto como argumentos para \texttt{-{}-verify}:

\begin{verbatim}
$ gpg --verify rfc4880.txt.asc rfc4880.txt
\end{verbatim}

Uma saída do tipo:

\begin{verbatim}
gpg: Signature made Mon 22 Jul 2013 20:55:29 AM BRT using RSA key
ID BE7A9C8C
gpg: Good signature from "Alice <alice@example.com>"
\end{verbatim}

Indica que o arquivo realmente confere com aquele enviado por Alice.  Para
fazer um teste de integridade, pode-se abrir o arquivo \texttt{rfc4880.txt},
fazer qualquer alteração ---como a adição de uma letra--- e salvar.  Quando a
verificação for feita, a palavra \texttt{Good}, na segunda linha da saída, será
trocada por \texttt{BAD}, indicando que a assinatura não confere com o arquivo.
Isso significa que se Alice fizer qualquer alteração no arquivo após gerar a
assinatura, ela terá de refazer a mesma, sob pena de poder ter seu arquivo
rejeitado por Bob.


\section{Ataques}
\label{sec:ataques}

De acordo com \cite{gnupg:man}, a parte mais frágil do sistema criptográfico
inteiro é a senha do usuário.  Desta forma, bem mais simples do que atacar
fragilidades dos algoritmos criptográficos modernos ---como o AES---,  é tentar
descobrir a senha usada na criptografia.  Nesta seção serão apresentados dois
exemplos neste sentido, um para a criptografia simétrica e outro para
assimétrica.

\subsection{Criptografia Simétrica}
\label{sec:ataques:simetrica}

Objetivando testar a segurança de arquivos criptografados com algoritmos
simétricos, foi criado o programa listado na figura \ref{lst:gpgbf.py}.  O
mesmo realiza um ataque de força bruta simples contra um arquivo passado por
parâmetro, testando combinações de senhas de até 6 caracteres, com letras
minúsculas e números ---não inclui caracteres acentuados ou cedilha.  Para
realizar o primeiro teste, o arquivo \texttt{rfc4880.txt} foi criptografado
usando a senha \texttt{alice}.

\begin{verbatim}
$ gpg --symmetric --cipher-algo aes256 rfc4880.txt
\end{verbatim}

\begin{figure}
\caption{Programa para atacar arquivos criptografados com o GnuPG.}
\label{lst:gpgbf.py}
    \lstinputlisting[lastline=59]{gpgbf.py}
\end{figure}
\begin{figure}
\caption{Continuação da figura \ref{lst:gpgbf.py}.}
    \lstinputlisting[firstline=60,firstnumber=60]{gpgbf.py}
\end{figure}

Tendo o arquivo criptografado e o \textit{script} ---arquivo
\texttt{gpgbf.py}--- no mesmo diretório, pode-se iniciar o ataque
com\footnote{A saída foi redirecionada para \texttt{/dev/null}, pois o foco do
teste foi o tempo para descoberta da senha.  Em um ambiente normal, o comando
\texttt{time} e esse redirecionamento podem ser omitidos.}:

\begin{verbatim}
$ time python gpgbf.py rfc4880.txt.gpg > /dev/null
\end{verbatim}

Nos testes realizados\footnote{Para cada exemplo foram feitos três testes e o
resultado foi a média deles.}, em um MacBook 4.1 MB404*/A ---processador
Intel{\footnotesize \textsuperscript{\textregistered}} Core{\footnotesize
\textsuperscript{\texttrademark}} 2 Duo e 2 GB de memória--- rodando Ubuntu
12.04\footnote{Sistema com todas as atualizações instaladas até o momento de
execução dos testes.  Durante este período, havia outros processos rodando, que
devem ter afetado o desempenho dos programas.}, levou-se mais de 30 minutos
para quebrar a senha ---os testes foram interrompidos após este tempo.

Para o segundo teste, o arquivo foi criptografado novamente usando-se a senha
\texttt{bob}.  Neste caso, com uma senha mais fraca ainda, levou-se $\approx$
32,15 segundos para descobrir a chave.

\subsection{Criptografia Assimétrica}
\label{sec:ataques:assimetrica}

Uma característica das chaves privadas é que elas precisam de uma senha
---definida durante a sua criação--- para serem utilizadas.  Para isso não
basta que seu utilizador possua o arquivo da chave: ele precisa saber a senha
utilizada nela.

Imaginando um cenário onde se possua a chave privada, mas não se conheça a
senha, pode-se lançar mão do John the
Ripper\footnote{\url{http://www.openwall.com/john}} (John) para realizar
ataques de força bruta ou de dicionário sobre a chave privada.

John é, segundo \cite{john}, um rápido quebrador de senhas para diversos
sistemas operacionais, cujo objetivo primário é de detectar senhas fracas no
Unix.  Por se tratar de um software livre, qualquer pessoa com conhecimento
suficiente pode alterar o seu código-fonte e liberá-lo, criando um
\textit{fork} do projeto.  Um desses \textit{forks} possui um componente que
torna possível realizar ataques sobre chaves privadas e pode ser baixado em seu
repositório
oficial\footnote{\url{https://github.com/magnumripper/JohnTheRipper}}.

Após o download, basta descompactar o arquivo, acessar o diretório criado e
consultar o arquivo \texttt{docs/INSTALL} para instruções sobre a instalação.

Uma vez que o John esteja pronto para uso, pode-se usar a chave privada de
Alice ---criada na seção \ref{sec:pratica:assimetrica}---, que utiliza uma
senha fraca ---\texttt{alice}--- para testar o funcionamento do programa.

Primeiramente deve-se exportar a chave privada de Alice, de acordo com a seção
\ref{sec:pratica:assimetrica}.  Feito isso, é necessário usar o componente
\texttt{gpg2john} para transformar a chave em um hash aceitável pelo John:

\begin{verbatim}
$ run/gpg2john alice-seckey.asc > alice-seckey-hash.asc
\end{verbatim}

Obtido o hash ---arquivo \texttt{alice-seckey-hash.asc}---, basta aplicar o
John sobre ele e aguardar o retorno do programa:

\begin{verbatim}
$ run/john alice-seckey-hash.asc
\end{verbatim}

Nos dois exemplos foram necessários $\approx$ 0,35 segundos\footnote{Entre cada
execução, o arquivo \texttt{run/john.pot} foi apagado, pois ele é um
\textit{cache} de senhas descobertas, para agilizar futuras operações sobre o
mesmo arquivo} para quebrar as senhas.

É interessante notar que o John permite ataques baseados em dicionários.  Para
isso deve-se criar um arquivo texto em branco ---e.g., \texttt{alice-
dict.lst}--- e colocar uma tentativa de senha por linha.  Usa-se então o
parâmetro \texttt{-{}-wordlist} para indicar o arquivo de dicionário:

\begin{verbatim}
$ run/john --wordlist=alice-dict.lst alice-seckey-hash.asc
\end{verbatim}

No caso de ataques de dicionário, caso a senha esteja no arquivo, o tempo para
descoberta é instantâneo ---principalmente porque as senhas usadas nos exemplos
são candidatas para vários dicionários\footnote{O John vem com um arquivo
---\texttt{run/password.lst}--- que é um dicionário básico de senhas.  A título
de curiosidade, ele possui as senhas dos dois exemplos cadastradas.}  Na
Internet há vários arquivos com dicionários prontos para uso.  A OpenWall
---mantenedora do John---, por exemplo, comercializa alguns deles, mas muitos
podem ser encontrados de graça na Internet.


\section{Conclusão}
\label{sec:conclusao}
A criptografia provê uma camada de segurança extra às comunicações eletrônicas.
No caso da interceptação de uma mensagem criptografada por uma terceira parte,
não é trivial a obtenção do conteúdo decriptografado sem o conhecimento das
chaves criptográficas.  Ainda assim, como no dito popular, \textit{``uma
corrente é tão forte quanto seu elo mais fraco''}, de nada adianta utilizar
métodos criptográficos de última geração, se não houver cuidado com o
gerenciamento das chaves.

Através dos exemplos apresentados na seção \ref{sec:ataques}, pode-se perceber
que o tamanho da senha é, assim como a diversidade de caracteres ---e.g.,
letras maiúsculas, minúsculas, números e caracteres especiais--- que compõem a
mesma, fator decisivo para segurança em ataques de força bruta.  Contudo, uma
senha longa e com diferentes tipos de caracteres, poderá ser facilmente
descoberta por um ataque de dicionário, caso seja uma palavra ou expressão de
uso comum.

É importante, portanto, que haja uma política bem definida para as senhas
usadas nos sistemas de criptografia.  Os usuários devem estar cientes sobre a
utilização de senhas fortes ---que utilizem conjuntos de caracteres diferentes
e que não sejam suscetíveis a ataques de dicionário--- e sobre a importância de
mantê-las em sigilo.  Além disso, os administradores de sistema devem
atentar-se para a correta configuração da infraestrutura, de forma que dados
cruciais para a seguraça ---como senhas e chaves privadas--- trafeguem de forma
segura pela rede e tenham seu acesso restrito dentro de uma organização.

Desta forma, pode-se perceber que a educação das pessoas, no tangente à
segurança da informação e à correta configuração do ambiente, é, somado à
utilização de técnicas criptográficas, fator primordial para manutenção da
segurança a níveis aceitáveis.


\section{Propostas para Trabalhos Futuros}
\label{sec:propostas}
Apesar do programa apresentado como exemplo na figura \ref{lst:gpgbf.py} ser
apenas ilustrativo, pode-se melhorá-lo, fazendo com que seja capaz de, por
exemplo, processar um conjunto maior de caracteres, que priorize palavras de
uso comum e aceite sugestões do usuário, como o parâmetro \texttt{-{}-wordlist}
do John.  Além disso, pode-se reescrevê-lo usando \textit{threads}, como forma
de melhor seu desempenho.

Seria interessante ainda, fazer um estudo mais aprofundado sobre as classes de
algoritmos ---simétricos e assimétricos---, sobre os algoritmos apresentados ou
mesmo apresentar outras soluções e seus usos.  Igualmente relevante seria a
escolha de outra solução similar ao GnuPG, para mostrar como ela trata o
processo de criptografia e o gerenciamento das chaves.  Ainda sobre o GnuPG,
poder-se-ia apresentar ou desenvolver uma interface gráfica para o mesmo, como
forma de mostrar as facilidades que ela traria para o processo de criptografia
e o gerenciamento de chaves criptográficas.


%%
% BIBLIOGRAPHY
%
\bibliographystyle{apalike}
\bibliography{fosforos}

\end{document}
