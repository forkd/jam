% Hevea - Curriculum Vitae (CV)
% CV template.
%
% Author: José Lopes de Oliveira Jr. <jilo.cc>
%
% LICENSE
% This program is free software: you can redistribute it and/or modify
% it under the terms of the GNU General Public License as published by
% the Free Software Foundation, either version 3 of the License, or
% (at your option) any later version.
%
% This program is distributed in the hope that it will be useful,
% but WITHOUT ANY WARRANTY; without even the implied warranty of
% MERCHANTABILITY or FITNESS FOR A PARTICULAR PURPOSE.  See the
% GNU General Public License for more details.
%
% You should have received a copy of the GNU General Public License
% along with this program.  If not, see <http://www.gnu.org/licenses/>.
%%


%%
% PREAMBLE
% Size: 10pt; 11pt; 12pt.
% Paper: a4paper; letterpaper; a5paper; legalpaper; executivepaper; landscape.
% Font: sans; roman.
\documentclass[11pt,a4paper,sans]{moderncv}
\usepackage[utf8]{inputenc}
\usepackage[T1]{fontenc}
\usepackage[brazil]{babel}
\usepackage[scale=0.75]{geometry}  % custom page margins

% Styles: casual; classic; oldstyle; banking.
% Color schemes: blue; black; grey; green; orange; purple; red.
\moderncvstyle{classic}
\moderncvcolor{black}


%%
% HEADER DATA
%
\name{José Lopes}{Oliveira Jr.}
\title{Analista de Segurança da Informação}
\address{Lorem Ipsum Dolor, 666, Sit Amet}{Apto. 103, 
CEP 30000-000}{Belo Horizonte, MG, Brasil}
\phone{31 9999-9999}
\email{email@jilo.cc}
\homepage{jilo.cc}
\social[linkedin]{forkd}
\social[github]{forkd}


%%
% THE DOCUMENT
%
\begin{document}
\makecvtitle


\section{Objetivo}
Elaborar e manter a política de segurança da informação do ambiente 
tecnológico e para a continuidade dos negócios, prospectar soluções de 
segurança da informação e comunicações, realizar a análise dos recursos 
técnicos para possível implantação, monitorar o tráfego e acesso aos 
ambientes computacionais, identificar e orientar quanto ao tratamento de 
vulnerabilidades, avaliar notificações de alertas emitidos pelos diversos 
órgãos de segurança, tratar ocorrências reportadas e identificadas em 
processos investigativos por meio de análise de trilhas de auditoria, bem 
como elaborar relatórios para atender solicitações de órgãos externos.


\section{Formação}
%\cventry{}{}{}{}{}{}
\cventry{2006--2009}{Especialista em Administração em Redes 
Linux}{Universidade Federal de Lavras}{Lavras (MG)}{}{}

\cventry{2004--2005}{Técnico em Informática}{IF Sudeste de Minas 
Gerais}{Barbacena (MG)}{}{}

\cventry{2002--2005}{Bacharel em Ciência da Computação}{Universidade 
Presidente Antônio Carlos}{Barbacena (MG)}{}{}


\section{Experiência}
\cventry{2013}{Analista de Segurança da Informação}{Cemig}{}{}{Revisão do 
processo de Gestão de Risco de Segurança da Informação; implementação do 
CVSS para metrificação de vulnerabilidades; modelagem do processo de 
Segurança da Informação ITIL\textsuperscript{\textregistered}; atuação no 
projeto do Portal de Assinaturas, visando \textit{paper less} com 
certificados digitais; participação em projetos de segurança cibernética 
voltados para \textit{smart grids}; criação e revisão de instruções 
técnicas de segurança; experiência no gerenciamento de incidentes em todo 
seu ciclo de vida ---descoberta, registro, tratamento e verificação---; 
criação do Grupo de Resposta a Incidentes de Segurança da Informação 
(CSIRT); cálculo do Índice de Segurança da Informação usando a ferramenta 
Risk Manager; participação no projeto de BYOD.}

\cventry{2012--2013}{Analista de Infraestrutura}{Secretaria Estadual de 
Meio Ambiente e Desenvolvimento Sustentável}{Belo Horizonte 
(MG)}{}{Administração de firewall com iptables. Operação de blades e 
storage HP.  Gestão ---criação, remoção e dimensionamento--- de máquinas 
virtuais em Citrix.  Mapeamento e documentação do parque de servidores.  
Instalação do servidor de mensagens instantâneas OpenFire.  Gestão em 
servidores Windows 2003 e Linux Debian e CentOS.  Responsável pela 
instalação, configuração e monitoramento de servidores com Zabbix e IPPlan.}

\cventry{2010--2012}{Desenvolvedor Web/Administrador de 
Sistemas}{sociativa Tecnologia}{Barbacena (MG)}{}{Desenvolvimento da 
política e implementação do sistema automatizado de backup do servidor Web 
---Debian Linux.  Manutenção e configuração dos produtos Google Apps 
---Analytics, Email, Docs e Agenda.  Responsável pela atualização do serviço 
de DNS e configuração do Apache através de arquivos .htaccess.  
Configuração de clientes Filezilla com SFTP.}

\cventry{2007--2012}{Analista de Suporte}{Pró-Renal Centro de Nefrologia 
Ltda.}{Barbacena (MG)}{}{Criação do ``Setor de Informática e Informação''.  
Gerenciamento da rede de computadores; manutenção do parque computacional; 
e criação e aplicação da Política de Uso e Segurança dos Recursos 
Computacionais.  Implantação do serviço de Voz sobre IP (VoIP) usando 
operadora terceirizada.  Migração para o novo software de gestão das 
sessões de hemodiálise e migração para BrOffice e Firefox.  Configuração de 
máquinas virtuais com VirtualBox em clientes OS X e suporte em sistemas 
Macintosh.  Responsável pelo gerenciamento dos projetos do novo anfiteatro 
e do ``Guia do Portador de Doença Renal''.}


\section{Qualificações}
\cventry{Inglês}{Avançado}{Instituição: Fisk}{Belo Horizonte (MG)}{}{} 

\cventry{2012}{Governança de TI Adotando o Modelo do 
Cobit\textsuperscript{\textregistered}}{TIExames Consultoria e 
Treinamentos}{Barbacena (MG)}{16 horas}{} 

\cventry{2011}{Fundamentos da Segurança da Informação com Base na ISO/IEC 
27002}{TIExames Consultoria e Treinamentos}{Barbacena (MG)}{10 horas}{} 

\cventry{2011}{ISO/IEC 27001 -- Sistema de Gestão de Segurança da 
Informação}{TIExames Consultoria e Treinamentos}{Barbacena (MG)}{10 
horas}{} 

\cventry{2005}{Curso de Extensão em Linux}{Universidade Presidente Antônio 
Carlos}{Barbacena (MG)}{60 horas}{}

\cventry{2003}{Assitência Técnica em Microcomputadores}{Talent 
Desenvolvimento e Treinamentos Técnicos}{Barbacena (MG)}{49 horas}{} 

\cventry{2000}{Introdução e Operação à Informática}{Sesi MG}{Barbacena 
(MG)}{80 horas}{} 


\section{Certificações}
\cvitemwithcomment{2012}{Cobit\textsuperscript{\textregistered} 
Foundation}{ISACA}

\cvitemwithcomment{2011}{ISFS (ISO/IEC 27002)}{EXIN}

\cvitemwithcomment{2010}{ITIL\textsuperscript{\textregistered} 
Foundation}{EXIN}


\section{Produções Acadêmicas}
\cventry{2009}{Windows Registry Fixer (WRF): Automação na Criação de 
Scripts para Controle de Permissões do Windows através do 
Samba}{Universidade Federal de 
Lavras}{\url{http://www.ginux.ufla.br/node/295}}{}{}

\cventry{2005}{Estudo Comparativo entre Sistemas Gerenciadores de 
Conteúdo}{}{}{Universidade Presidente Antônio Carlos}{}

\cventry{2005}{Drupal - Sistema Gerenciador de Conteúdo}{IF Sudeste de 
Minas Gerais}{}{}{}


\section{Habilidades em Informática}
\cvdoubleitem{Básico}{C, Javascript, jQuery, Bootstrap, SQL, 
HP EVA, nmap}{Avançado}{Shell Script}

\cvdoubleitem{Intermediário}{iptables, Python, Regex, \LaTeX2e, Git, 
Risk Manager}{Vários}{Linux, Apache, WordPress, Zabbix, IPPlan, Citrix XenServer}


\section{Principais Projetos em Software Livre}
\cventry{Javascript}{Abacus}{Calculadora CVSS v2 de Código 
Aberto}{\url{https://github.com/forkd/abacus}}{}{}

\cventry{Javascript}{Bradocs}{Conjunto de funções para validação de alguns 
documentos brasileiros}{\url{https://github.com/forkd/bradocs}}{}{}

\cventry{Shell Script}{Narkissos}{Tratamento em lote de 
imagens}{\url{https://github.com/forkd/narkissos}}{}{}

\cventry{Shell Script}{Weback}{Ferramenta genérica de backup ---arquivos e 
bases MySQL---, com espelhamento no 
Dropbox}{\url{https://github.com/forkd/weback}}{}{}


\end{document}
